\documentclass[UKenglish]{ifimaster}  %% ... or USenglish or norsk or nynorsk
\usepackage[latin1]{inputenc}         %% ... or utf8 or applemac
\usepackage[T1]{fontenc,url}
\urlstyle{sf}
\usepackage{babel,textcomp,csquotes,ifimasterforside,varioref,graphicx}
\setlength{\parskip}{\baselineskip}%
\setlength{\parindent}{0pt}%
\title{Do Work in Progress (WIP) - Limit in Agile Software Development Matter?}        %% ... or whatever
\author{Truls Skeie}                      %% ... or whoever 
\usepackage[
    backend=biber,
    style=authoryear-icomp,
    sortlocale=de_DE,
    natbib=true,
    url=false, 
    doi=true,
    eprint=false
]{biblatex}
% !BIB TS-program = biber

\addbibresource{lib.bib}

\begin{document}
\ififorside{}
\frontmatter{}
\maketitle{}

\chapter*{Abstract}                   %% ... or Sammendrag or Samandrag

\tableofcontents{}
\listoffigures{}
\listoftables{}

\chapter*{Preface}                    %% ... or Forord

\mainmatter{}
\part{Introduction}                   %% ... or Innledning or Innleiing


\chapter{Background}                  %% ... or Bakgrunn
In the field of WIP and if WIP matters in software development, lacks proper research, but Giulio Concas and Hongyu Zhang has done research on the difference between limit WIP and unlimited WIP  \parencite{SMR:SMR1599}.

Work in progress is a tool that helps teams limits their tasks by specifying how many tasks a developer can be assigned to at once. WIP helps team to reduce overhead, decrease leadtime and increase throughput %%finne reff

How to find the best WIP in a given interval and context also lacks proper research, but in manufacture business some research has been done. Taho Yanga, Hsin-Pin Fub, Kuang-Yi Yanga stated that WIP could be defined as; WIP = cycle time * throughput rate in manufacture business \parencite{Yang}.

\section{Kanban}
''We can define Kanban software process as a WIP limited pull system visualized by the Kanban board''  \parencite{DavidAnderson}.

Kanban system focus on;
\begin{itemize}
\item Continuous flow of work
\item	No fixed iterations or sprints
\item Work is delivered when it?s done
\item Teams only work on few tasks at the time specified by the limit WIP
\item Make constant flow of released tasks
\end{itemize}
\parencite{DavidAnderson}.

Toyota production system was introduced Kanban during late 1940s and early 1950s in order to catch up with the American car industry \parencite{ono1988toyota} . In the last ten years software Development Company have started to implement agile methods and Kanban is one of them \parencite{Conboy}.  Kanban splits one big problem into many small pieces of problems.

When the small pieces are in order, the problems are put up on the Kanban-board to visualize the problems and see potential bottlenecks. When people started to understand Kanban, they easily discovered where the bottlenecks where, and started to help where the bottlenecks where \parencite{Shinkle}.

One of the most important people in Kanban software developmen, David Anderson  also referred to as ''father of Kanban in the software development industry''  \parencite{InfoQ:2013:May:Online}, author of Kanban: Successful Evolutionary Change for Your Technology Business once stated ''If you think that there was Capability Maturity Model Integration, there was Rational Unified Process, there was Extreme Programming and there was Scrum, Kanban is the next thing in that succession.''   \parencite{InfoQ} 

More and more software projects adapt to Kanban, and this is one of the reasons why this thesis will focuses on Kanban and WIP.
Kanban is one of the agile method in the wind these days, and is used with Lean Software development which is one of the fastest growing approaches in software development \parencite{DavidAnderson}

One of the main difference between Scrum and Kanban is estimation, in simulation of software maintenance process, with and without a work-in-process limit \parencite{SMR:SMR1599} estimation was defined to be the main source of waste. In their research, they find out, if they let the developers work with small tasks at time and not be interrupted, they will be more effective. The developers in this case was interrupted when they was assigned to estimate tasks. The research groups decided to implement Lean-Kanban, which includes minimalizing waste, which meant estimation for this case. After implementing Lean-Kanban the team?s increased the ability to perform work, lower the lead time and meet the production dates.

\section {Scrum}
Scrum \parencite{Scrum}
\part{The project}                    %% ... or ??

\chapter{Planning the project}        %% ... or ??


\part{Conclusion}                     %% ... or Konklusjon

\chapter{Results}                     %% ... or ??


\backmatter{}
\printbibliography
\end{document}
