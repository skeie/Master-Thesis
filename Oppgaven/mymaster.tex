\documentclass[UKenglish]{ifimaster}  %% ... or USenglish or norsk or nynorsk
\usepackage[latin1]{inputenc}         %% ... or utf8 or applemac
\usepackage[T1]{fontenc,url}
\urlstyle{sf}
\usepackage{babel,textcomp,csquotes,ifimasterforside,varioref,graphicx}
\setlength{\parskip}{\baselineskip}%
\setlength{\parindent}{0pt}%
\title{Do Work in Progress (WIP) - Limit in Agile Software Development Matter?}        %% ... or whatever
\author{Truls Skeie}                      %% ... or whoever 
\usepackage[
    backend=biber,
    style=authoryear-icomp,
    sortlocale=de_DE,
    natbib=true,
    url=false, 
    doi=true,
    eprint=false
]{biblatex}
% !BIB TS-program = biber

\addbibresource{lib.bib}

\begin{document}
\ififorside{}
\frontmatter{}
\maketitle{}

\chapter*{Abstract}                   %% ... or Sammendrag or Samandrag

\tableofcontents{}
\listoffigures{}
\listoftables{}

\chapter*{Preface}                    %% ... or Forord

\mainmatter{}
\part{Introduction}                   %% ... or Innledning or Innleiing
In the field of WIP and if WIP matters in software development, lacks proper research, but Giulio Concas and Hongyu Zhang has done research on the difference between limit WIP and unlimited WIP  \parencite{SMR:SMR1599}.

Work in progress is a tool that helps teams limits their tasks by specifying how many tasks a developer can be assigned to at once. WIP helps team to reduce overhead, decrease leadtime and increase throughput %%finne reff

How to find the best WIP in a given interval and context also lacks proper research, but in manufacture business some research has been done. Taho Yanga, Hsin-Pin Fub, Kuang-Yi Yanga stated that WIP could be defined as; WIP = cycle time * throughput rate in manufacture business \parencite{Yang}.

\chapter{Background}                  %% ... or Bakgrunn
''We can define Kanban software process as a WIP limited pull system visualized by the Kanban board''  \parencite{DavidAnderson}.

\part{The project}                    %% ... or ??

\chapter{Planning the project}        %% ... or ??


\part{Conclusion}                     %% ... or Konklusjon

\chapter{Results}                     %% ... or ??


\backmatter{}
\printbibliography
\end{document}
