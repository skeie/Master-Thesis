\documentclass[UKenglish]{ifimaster}  %% ... or USenglish or norsk or nynorsk
\usepackage[latin1]{inputenc}        %% ... or utf8 or applemac
\usepackage[T1]{fontenc,url}
\urlstyle{sf}
\usepackage{babel,textcomp,csquotes,ifimasterforside,varioref,graphicx,caption,nameref,caption}
\usepackage[section]{placeins}
\usepackage{amsmath,lipsum}
\setcounter{secnumdepth}{3}
\interfootnotelinepenalty=10000
\usepackage{listings}
\usepackage{color}
\usepackage{array, dcolumn, booktabs,enumitem}
\usepackage[showframe=false]{geometry}
\usepackage{changepage}
\newcommand{\mypm}{\mathbin{\smash{%
\raisebox{0.35ex}{%
            $\underset{\raisebox{0.5ex}{$\smash -$}}{\smash+}$%
            }%
        }%
    }%
}


\definecolor{mygreen}{rgb}{0,0.6,0}
\definecolor{mygray}{rgb}{0.5,0.5,0.5}
\definecolor{mymauve}{rgb}{0.58,0,0.82}

\lstset{ %
  backgroundcolor=\color{white},   % choose the background color; you must add \usepackage{color} or \usepackage{xcolor}
  basicstyle=\footnotesize,        % the size of the fonts that are used for the code
  breakatwhitespace=false,         % sets if automatic breaks should only happen at whitespace
  breaklines=true,                 % sets automatic line breaking
  captionpos=b,                    % sets the caption-position to bottom
  commentstyle=\color{mygreen},    % comment style
  deletekeywords={...},            % if you want to delete keywords from the given language
  escapeinside={\%*}{*)},          % if you want to add LaTeX within your code
  extendedchars=true,              % lets you use non-ASCII characters; for 8-bits encodings only, does not work with UTF-8
  frame=single,                    % adds a frame around the code
  keepspaces=true,                 % keeps spaces in text, useful for keeping indentation of code (possibly needs columns=flexible)
  keywordstyle=\color{blue},       % keyword style
  language=Java,                 % the language of the code
  morekeywords={*,...},            % if you want to add more keywords to the set
  numbers=left,                    % where to put the line-numbers; possible values are (none, left, right)
  numbersep=5pt,                   % how far the line-numbers are from the code
  rulecolor=\color{blue},         % if not set, the frame-color may be changed on line-breaks within not-black text (e.g. comments (green here))
  showspaces=false,                % show spaces everywhere adding particular underscores; it overrides 'showstringspaces'
  showstringspaces=false,          % underline spaces within strings only
  showtabs=false,                  % show tabs within strings adding particular underscores
  stepnumber=1,                    % the step between two line-numbers. If it's 1, each line will be numbered
  stringstyle=\color{mymauve},     % string literal style
  tabsize=2,                       % sets default tabsize to 2 spaces
  title=\lstname,                   % show the filename of files included with \lstinputlisting; also try caption instead of title
  numberstyle=\tiny,
}





\setlength{\parskip}{\baselineskip}%
\setlength{\parindent}{0pt}%


\title{Do Work in Progress (WIP) - Limit in Agile Software Development Matter?}        %% ... or whatever
\author{Truls Skeie}                      %% ... or whoever 
\usepackage[
    backend=biber,
    style=authoryear-icomp,
    sortlocale=de_DE,
    natbib=true,
    url=false, 
    doi=true,
    eprint=false,
    citestyle=authoryear
]{biblatex}
% !BIB TS-program = biber

\addbibresource{lib.bib}

\begin{document}
\ififorside{}
\frontmatter{}
\maketitle{}

\chapter*{Abstract}                   %% ... or Sammendrag or Samandrag

\tableofcontents{}
\listoffigures{}
\listoftables{}

\chapter*{Preface}                    %% ... or Forord

\mainmatter{}
\part{Introduction}                   %% ... or Innledning or Innleiing
\chapter{Background}  
\label{chap:Background}                %% ... or Bakgrunn
In this master thesis the main topics will contain analyzing of the gathered data from Software Innovation (SI) in order to see if WIP - limit in agile methods matters. SI is a Scandinavian software company that delivers Enterprise Content Management applications. From 2008 to 2013 SI gathered information about each task that was develop. The main reason SI gathered the data was to see if Kanban was more sufficient than Scrum for their use. For interested reader, the case study can be found in the article "Quantifying the Effect of Using Kanban versus Scrum: A Case Study" \parencite{Dag}. In this thesis the data will be used to determine if WIP-limit matters in agile methods.  

In the this chapter there will be a brief introduction of Scrum, Kanban, affiliated tools and to Software Innovation.
\newpage
\section{Kanban}
''We can define Kanban software process as a WIP limited pull system visualized by the Kanban board''  \parencite{DavidAnderson}.

Toyota production system introduced Kanban as a scheduling system for lean and just-in-time (JIT) production during late 1940's and early 1950's in order to catch up with the American car industry. The Kanban method combined with the lean approach was a success for Toyota. The success was noticed by the software development industry among other. In the last ten years more and more,  software development companies have started to implement agile methods such as Scrum and Kanban \parencite{Conboy}, \parencite{ono1988toyota}. 

More and more software projects adapt to Kanban, and this is one of the reasons why this thesis will focuses on Kanban and one of it's main tools WIP.
Kanban is one of the agile method in the wind these days, and is used with Lean Software development which is one of the fastest growing approaches in software development \parencite{DavidAnderson}
One of the most important people in Kanban software development, David Anderson  also referred to as ''father of Kanban in the software development industry''  \parencite{InfoQ:2013:May:Online} and author of the book "Kanban: Successful Evolutionary Change for Your Technology Business" once stated ''If you think that there was Capability Maturity Model Integration, there was Rational Unified Process, there was Extreme Programming and there was Scrum, Kanban is the next thing in that succession.''   \parencite{InfoQ} 

The Kanban method splits one big problem into many small pieces of problems. When the small pieces are defined by the team, the problems are put up on the Kanban-board to visualize the problems and see potential bottlenecks during development. When people start to understand Kanban, they easily discovered where the bottlenecks are \parencite{Shinkle}.

Kanban system focus on:
\begin{itemize}
\item Continuous flow of work
\item	No fixed iterations or sprints
\item Work is delivered when it's done
\item Teams only work on few tasks at the time specified by the WIP limit
\item Make constant flow of released tasks
\parencite{DavidAnderson}.
\end{itemize}
\newpage

One of the main difference between Scrum and Kanban is estimation, in simulation of software maintenance process, with and without a work-in-process limit \parencite{SMR:SMR1599} estimation was defined to be the main source of waste. In their research, they find out, if they let the developers work with small tasks at time and not be interrupted, they will be more effective. The developers in this case was interrupted when they was assigned to estimate tasks. The research groups decided to implement Lean-Kanban, which includes minimizing waste, which meant estimation for this case. After implementing Lean-Kanban the teams increased the ability to perform work, lower the lead time and meet the production dates.


\section {Lean-Kanban}
The Lean approach was introduced between 1948 and 1975 in manufacturing work in Japan. It was designed to find and eliminate waste, so the manufacturing could deliver value to the costumer more efficiently. In 2003 Mary and Tom Poppendieck first introduced Lean thinking to software development, they published 'Principles of Lean Thinking' \parencite{Lean:2003}. Poppendieck stated that an important tool to manage work flow is the concept of pull-systems, which means tasks are put in production only when a costumer asks for it \parencite{Lean:2009}.
In the recent years, Kanban has been introduced more and more to software development, and is becoming one of the keys to Lean practice in the field \parencite{DavidAnderson}. 

\section {Scrum}
The Scrum framework is the source of much of the thinking behind principles and values of the Agile Manifesto. Values as "Individuals and interactions over processes and tools", "Working software over comprehensive documentation", "Customer collaboration over contract negotiation" and "Responding to change over following a plan" relates directly to Scrum. \parencite{Scrum}.

Scrum have three main roles, the Product Owner, the Scrum Master and the members of the development team. The Product owner in collaboration with the Scrum Master decides which work to be prioritized in the backlog. A backlog represents the tasks to be done in order to complete the project. The Scrum Master acts like a team leader and helps the team and organization to take best advantages of Scrum. The development team works on tasks specific for the sprint there in \parencite{Scrum}.

Sprint is a time-boxed interval over a given time. The Scrum framework suggests the duration of sprints to be from one to four weeks. Before each sprint, a sprint planning meeting is conducted, with all the team members attending.  A Sprint planning meeting is held so the team can discuss tasks from the backlog and come to an agreement of which tasks to be put in the minimal backlog.  \parencite{Scrum}.

In each sprint a minimal backlog is created so the developer knows which tasks to work on. The Product Owner and the team members discuss and decide which tasks from the backlog to be added to the minimal backlog. After the minimal backlog is full, the Product Owner and the team members discuss each task in order to get a better and a shared understanding of what is required in order to complete the task. One of the main principles in Scrum is that it requires that a new feature is ready for release after a sprint. The feature should be a visible part of the product in order to get feedback from end-users. So all the tasks in the minimal backlog combined should be a visible of the product.  \parencite{Scrum}.

\section {Kanban Board}
''The Kanban board makes it clear to all the team members the exact status of progress, blockages, bottlenecks and they also signal possible future issues to prepare for''\parencite{Joyce}.

The Kanban board is one of many important tools in Kanban. It's used to control the WIP, increase the information flow with visualization and spot bottlenecks \parencite{SMR:SMR1599}. A Kanban is board is illustrated in figure \ref{kanban_board}. Each column has an intuitive name in order to describe itself so the developers easily can track where each task is. In figure \ref{kanban_board} each column is named "Backlog", "In progress" and "Done".  Each column can have a WIP-limit to specify how many works in progress there are allowed in the column \parencite{Joyce}. In figure \ref{kanban_board} the WIP-limit is stated under the column name. The backlog columns have a WIP-limit of 4, In progress has 5 and obviously done doesn't have a WIP-limit. The yellow stickers represents the tasks. Some follow to path to mark stickers with different colors representing the severities. In "Kanban Implementation in a Telecom Product Maintenance" the stickers has three different colors, green, yellow and red. Each of the color indicates how close to overdue the tasks are. If the sticker is red, the tasks is already overdue \parencite{6068363}. 
\begin{figure}[ht!]
\centering
\includegraphics[width=90mm]{Picture/kanban_board.jpg}
\caption{Example of a Kanban board}
\label{kanban_board}
\end{figure}

\section {Lead time}
''Lead time is the total elapsed time from when a customer requests software to when the finished software is released to the customer'' \parencite{Joyce}.

Lead time is an essential ingredient when you look for the optimal WIP. Often in project, lead time is split into pieces, so every task has its own lead-time; this gives the development teams the advantages to experiment with different WIP's in order to see the different lead-times and then measure which WIP that suits this project the best. 

The citation by Joyce above is close to definition of what lead time is. This definition could be useful for consultancy companies, but for in-house development company with few releases each year this definition is unsuitable. Quantifying the effect of Using Kanban versus Scrum stated two reason why this is unsuitable for in house development companies: 

"First, the amount of time a work item remains in the backlog queue before it's put on the board is a function of priority, not whether the company uses Scrum, Kanban, or other development methods. Furthermore, companies that develop and sell products to many customers might propose new features themselves and put them on the backlog before any customers request them. Second, given a policy of two or three releases a year, the result of a work item isn't delivered to the customer immediately after it's finished'' \parencite{Dag}.


\section{Just-In-Time}
"Just-In-Time is based on delivering only the necessary products, to the necessary time and the necessary quantity." \parencite{JIT}.

Just-In-Time was introduced 30 years ago by Toyota Motor in combination with Lean.  JIT has been developed to increase productivity through waste reduction and increasing the value added on the production processes. In one of the books by Mary and Tom Poppendieck the JIT principle is explained \parencite{Lean:2006}. To explain, illustrate and visualize the JIT principle Mary and Tom Poppendieck uses the picture \ref{JITE}  \parencite{JIT} \parencite{Lean:2006}.

\begin{figure}[ht!]
\centering
\includegraphics[width=90mm]{Picture/JIT.jpg}
\caption{JIT example}
\label{JITE} % JIT example
\end{figure}


In the picture \ref{JITE} the stream reflects the inventory.  Under the stream, there are rocks located in different sizes. The rocks illustrates waste and problems that can occur.  If the stream level is lowered, the rocks are more visualized. At this point you have to clear out rocks in order to make the boat continue it's journey, or it will crash into the rocks. After the rocks are cleaned out, you can lower the stream level again and continue until it's just pebbles left

If one lower the stream, problem and waste will become visible. But why do Lean want to lower inventory in order to make problems and waste occur? Because when problem and waste occurs, you are able to fix the problem and remove the waste. Fixing the problem and removing the waste can have several benefits such as, your process could be optimized and you are on step closer to have zero problems and zero waste.  \parencite{JIT} \parencite{Lean:2006}.


\section{Throughput}
''The output of a production process (machine, workstation, line plant) per unit time (e.g., parts per hour) is defined as the systems throughput or sometimes throughput rate'' \parencite{Adams}

The main concept of throughput is to measure how productive teams, people or companies are. Throughput is measured in number of finished delivered tasks per hour, day, week, month, quarter or year. This applies to both software development and manufacturing. But there are some difference. In software development each task is more abstract than in manufacturing and in software development each tasks can have different solutions depending on how the team or developer approaches the task.
% and the tasks is usually only done once not mass produced.

In manufacturing each task usually has one solution and when the solution is found the physical item is mass produced.  Adam said "Throughput in plant, line or workstation, is defined as the average quantity of good  parts produced per unit time" \parencite{Adams}, which gives a good example of the relationship between tasks in manufacturing and software development. In manufacturing the part either fits its purpose (good) or not (defective). In software development a tasks can fit a purpose, but the purpose may be wrong. 
 
% As long as the software development task is bug free, it's delivered as non-defective but it may not fit the defined purpose by the end users. 

A key factor in successfully measuring throughput in software development is to specify the size of each task. If the size is not specified a developer x can have throughput of 1, but another developer could have throughput of 3 and they still have done the same amount of work and this will give wrong results if the throughput is measured.  So it's recommend to specify the amount of work for each task in advanced. 

\subsubsection{Example of throughput measurement}
This is a simple example to illustrate throughput with different task sizes. Team x had a throughput of eighteen tasks after the first quarter, twenty after the second, fifteen after the third and twelve after the last quarter and they used Scrum the first two quarters and Kanban the last two as illustrated by table \ref{tt}. It will look like team x benefits most from Scrum. But if the task during the Kanban time was twice the size of Scrum, Kanban would suite team x the best. In order to to get valid result from throughput measurement the size of tasks has to be agreed upon by the teams or company. 
\begin{table}[ht]
\begin{center}
    \begin{tabular}{| l | l | l | l |}
    \hline
    Quarter & Throughput & Method\\ \hline
    1 & 18 & Scrum\\ \hline
    2 & 20 & Scrum \\ \hline
    3 & 15 & Kanban\\ \hline
    4 & 12 & Kanban\\ \hline
    \end{tabular}
\caption{Throughput}
\label{tt} %% throughput table
\end{center}
\end{table}

\section{Churn}
\label{sec:Churn}
"Churn is defined as the sum of the number of lines added, deleted, and modified in the source code" \parencite{Dag}.

Churn is a measure that's not so quite known as lead time, throughput or WIP. Churn is a term that's used as surrogates for effort in software engineering. Many studies in software engineering use code churn or revisions as surrogate measure of effort \parencite{yamashita2012quantifying}. Emam stated that "analysts should be discouraged from using surrogate measures, such as code churn, unless there is evidence that they are indeed good surrogates \parencite{el2000methodology}."  The authors in "Quantifying the effect of code smells on maintenance effort" agree that one should cautious when using surrogates for effort. 

\section{WIP-limit}
\label{WIPsec}
''WIP-limits seem to be the worst understood part of the Kanban system. When used properly, it exposes bottlenecks and reduces lead time for individual work items. Used improperly, it can starve developers for work or result in too many people working on the same work items.'' \parencite{Shinkle}


WIP-limit is a tool in Kanban to reduce overhead by limit task-switching for each developer and visualize bottlenecks. One of the best way to explain WIP and the impact of WIP-limit is to use cars and roads as analogy. All roads have it maximum capacity of cars . When this limit is reached, traffic jam occurs and the throughput of cars decreases and lead time increases. The same can be said about software development teams, a software team has a maximum number of tasks they can perform, if the team is pushed over the maximum limit, the throughput of tasks decreases and lead time increases.


Research in the field of Work In Progress (WIP) and if WIP-limit matters in software development lacks proper research. But some research has been done, Giulio Concas and Hongyu Zhang for instance has done research on the difference between limit WIP and unlimited WIP\parencite{SMR:SMR1599} and David Anderson, Giulio Concas, Maria Ilaria Lunesu, and Michele Marchesi has also highlighted the difference between limit WIP and unlimited WIP the article 'Studying Lean-Kanban Approach Using Software Process Simulation' \parencite{DavidAnderson}. Lukasz proposes to use the effectiveness metric to help determine WIP-limit. The effectiveness metric should be applied after end cycle according to Lukasz, so after each cycle, one can apply it and the result could be used as a guideline for WIP-limit for the next cycle. The effectiveness takes the number of bugs found (ai) and the number of bugs found by external people (e.g. lawyers, accountants, coaches, consultants, translators, internal and external service providers ,etc.) (ei), and minus ai and ei, then divide the result by ai and multiply it by 100\%  as shown in \ref{WIPEQ} \parencite{Sienkiewicz}

\begin{equation} \label{WIPEQ}
Ei=\frac{ai-ei}{ai}*100\%
\end{equation}

In manufacture business Taho Yanga, Hsin-Pin Fub, Kuang-Yi Yanga stated that WIP could be defined as: 

\begin{equation} \label{WIPMAN}
WIP-limit = cycle time * throughput \: rate
\end{equation}
\parencite{Yang}.

When first implementing Kanban, Shinkle explains that the users don't care about WIP or setting a WIP-limit, but rather the visibility of Kanban through the Kanban board. When the user get more experience with Kanban, they start to attempt the principles of WIP-limit \parencite{Shinkle}. Srinivasan, Ebbing and Swearing said that setting the WIP-limit is not easy. They suggest that the WIP-limit is set, and then observe throughput, and adjust after that \parencite{Mandyam}. 

The principle of Kanban also says you should limit WIP, but what should the limit be? The principles of Kanban tells us to experiment \parencite{Kniberg}. Lean Software Management and the Impact of Kanban on Software Project Work suggest that WIP should be minimized as well. The study suggest to minimize WIP-limit to keep high quality \parencite{Ikonen} and to create continuous flow and bring problems to the surface \parencite{Joyce}. The conclusion of present study is to keep the WIP-limit low and experiment by slowly increase the WIP-limit until the throughput decreased and lead time increased, then you know that the previous WIP-limit was the perfect one.



\iffalse
Shinkle defined novice and more experience kanban-users with a descriptive analogy \parencite{Shinkle}:
''Think about a typical person wanting to bake a cake. They go to the store, purchase a boxed cake mix, and follow the directions as described on the back of the box. They have little to no knowledge about how to alter the recipe nor do they have a desire to do so. Their goal is simply to bake a cake.''

''An advanced beginner understands how to apply some context to the instructions or rules on the back of the box. They can make minor adjustments for things like altitude, pan size, oven conditions, etc. They are still following the basic recipe, but can make minor adjustments likely based on previous experiences." According to Shinkle, principles like WIP-limit will be adopted when the user has some experience with Kanban. 
\fi

\subsection{Benefits with WIP-limit according to various articles}
\begin{enumerate}
\item When lowering the WIP-limit will help people avoid task switching. When one is task swtiching it's hard to be able to fully concentrate. \parencite{Ikonen}.
\item There's stated when using short-cycle times and Kanban board to limit WIP, the software development team's learning is increased' \parencite{Joyce}
\item The team in Lean Software Management study realized the where bottlenecks. So the team started to determine WIP by their constraints. The team quickly realized that they had fewer quality assurance/testing staff and business analysts than software developers. This reflected the bottlenecks and the constraints, so the team adjusted the WIP-limit to how much work they could handle; this gave the team more experience in dealing with WIP and increased productivity \parencite{Joyce}.

\item To set a WIP-limit has some advantages:
\begin{itemize}[noitemsep,topsep=0pt,parsep=0pt,partopsep=0pt]
\item It helps team to reduce overhead 
\item Decrease lead time
\item Increase throughput
\item It reduces flow times
\item Reduces variation 
\item Improves quality
\end{itemize}
\parencite {CONWIP}.
\end{enumerate}

As one can see, several people have their opinion on WIP and the importance of defining WIP-limit. If we disregard Lukasz's effectiveness metric, there is no clear rule on how to determine the WIP-limit without  recording throughput and lead time and measure them, even though WIP is a crucial tool in order to use Kanban sufficient. 

\subsection {Limit WIP vs. Unlimited WIP}
Simulation of software maintenance process, with and without a work- in-process limit did a research on how the throughput and how developers experience WIP limit and unlimited WIP \parencite{SMR:SMR1599}.

One of the result from this paper was at the end of a simulation, the average of closed tasks was 4145 when the WIP was limited and 3853 when the limit was not limited (about 7\% less). The paper concludes their finds; developers are more focused on fixing few issues, because the number of issues they can work on is limited. The developers are more likely to continue on the issue from the day before, rather than starting on another issue, this reduce overhead. When developers start on a new issue, they need to use time to familiarize themselves with the code and the issue. That could create unnecessary overhead if some developer already has done it, but that developer is now working on a another issue. The study also showed that limit WIP can improve throughput and work efficiency. \parencite{SMR:SMR1599}.

\section{Software Innovation (SI)}
Software Innovation is a Scandinavian software company. SI develops and delivers market-leading Enterprise Content Management applications that helps organizations improve and increase efficiency in document management, case handling and technical document control. SI build products around Microsoft Sharepoint platform and tightly integrated into the Microsoft Office environment \parencite{Dag}. \parencite{SI}.

SI has approximately 300 employees. SI has offices in Oslo, Copenhagen and Stockholm and a development center in Bangalore\parencite{SI}.  From 2001 to 2006 SI used Waterfall process with a life cycle of design, implementation, testing, and deployment for each new release. In 2007 SI changed to Scrum. Scrum was implemented with the standard elements of Scrum: 
\begin{itemize}[noitemsep,topsep=0pt,parsep=0pt,partopsep=0pt]
\item Cross functional teams
\item Sprint planning meetings 
\item Estimation of work items using planning poker
\item Daily standup meetings
\item Sprints of three weeks
\end{itemize}
After a three week sprint a fully tested shippable code is ready. In 2010 SI went from Scrum to Kanban. SI felt that Scrum was too rigid and didn't fit their purpose, they also feared that inaccurate estimation and time boxes gave longer lead times. They also saw Scrum planning meetings as waste which reduced productivity and quality\parencite{SI}. 

SI decided to implemented Kanban in the following manner. When a work items starts SI tries to make the item flow through all the stages until its ready for release as quick as possible. In order for a item to be ready for release it has to be at a satisfactory quality level which is defined by SI. SI also implemented WIP limits, if the WIP limit is reached, no new tasks are started until another task is finished also known as just-in-time. \parencite{SI}.



%The data gathered for this theses is gathered used TFS. For this theses four teams will be analyzed to see if 



\part{The project}                    %% ... or ??
\chapter{Research Questions}
\label{chap:RQ}
In this thesis the overall research question will be to study the effects of WIP limits for SI. Hopefully the research will give answers to the following question:
\begin{itemize} 
\item See if the exist an optimal WIP-limit for a given context.
\item How to best find the optimal WIP-limit
\item Which parameters to consider in order to optimize WIP. 
\end{itemize}

\chapter{Research Methods}
\label{chap:RM}
In order to try answering the research questions stated in chapter \ref{chap:RQ} a dataset from SI will be analyzed. The set from SI contains programing tasks recorded from 2008 to 2013.  The data set is represented in a excel document.  The excel document contains various columns, but the columns of interest are listed in table \ref{IC}. Table \ref {dataset} shows an excerpt from the columns in the excel document. In table \ref{des} is a description of which columns the program needs in order to measure WIP, throughput, bugs and churn. Lead time for each task is already defined in the excel document, so lead time will be extracted from the data set.
 
Since there is no software for generating WIP, throughput, bugs and churn per day from the data set, a program was made. The program will transform the raw data from the excel document into quantitative data, which will be analyzed using SPSS. How the program will generate WIP will be described in section \ref{WPD},  bug in section \ref{Bug}, throughput in section \ref{TP} and churn in section \ref{churn}. The explanation of how the program works will be split into sections so it will be easier to get a understanding of how the program operates. The explanation of how the program works will first contain a detailed description in words, then a detailed description using Pseudocode\parencite{jd} will be provided. 

%%In this thesis the plan is to measure the data in one big step, where I measure all the teams and data in some given interval. I would also measure the data in small steps, when SI used Scrum, when SI was in the transforming phase and when they had changed to Kanban. Also I will measure WIP values for a given day, date, week, month, quarter and year. This will give me data to compare between teams in a specific interval.

%%I will reuse some data from the paper from Quantifying the Effect of Using Kanban vs. Scrum: A Case Study \parencite{Dag} Mainly I will look at churn and lead-time measured in the paper and compare them with WIP and throughput between teams. Hopefully the data measured will give me some sense of what WIP limit does for SI and which factors helping determine if WIP-limit matters in agile software development

%%Hopefully I would gather or get some other underlying data, so I will be able to compare my work, on the set from SI to the other data.
%%From the data created i will analyze them using SPSS.

%%The set from SI only contains dates when tasks was created, added to backlog, pulled from backlog or finished.  In order to measure average per month, quarter and year the program needs to know WIP and items in backlog for each day. In order to do so, the program will generate the remaining days.

\begin{table}[!ht]
\begin{center}
\begin{tabular}{| l | l | l | l | l | l | l |}
    \hline
    ID	& Type &  Created Date & From Day & Date To & Lead Time \\ \hline
    3027 & Bug & 2008-10-07 &  2008-10-09 & 2008-10-16 & 20\\ \hline
    3028 & Bug  & 2008-10-07 & 2008-10-07 & 2008-10-08 & 10\\ \hline
    3029 & Feature & 2008-10-07 &  2008-12-30	 & 2008-12-30 & 105\\ \hline
    3030 & Feature & 2008-10-07 & 2008-10-07	& 2008-10-07 & 1\\ \hline
    3035 & Bug & 2008-10-08 & 2008-11-20 & 2008-11-28 & 17\\ \hline
    3037 & Feature & 2008-10-08 &  2008-10-19	 & 2008-10-19 & 7\\ \hline
    3040 & Bug & 2008-10-10 &  2008-11-19 & 2008-11-19 & 48\\ \hline
    \end{tabular}
\caption{Excerpt from the dataset}
\label{dataset}
\end{center}
\end{table}
\newpage

\subsection{Information about how the measurement is done}
%%skrive om denne i inledningen?
\label{sec:information}
\begin{itemize}
\item The Date standard is specified as yyyy-mm-dd. 
\item When I write iterate it means looping through the data with a for each-loop. 
\item All seven days in the week are taken into account when measuring included Saturdays and Sundays
\item Quarter of a year is defined as: 
\begin{itemize}
\item January, February and March (Q1)
\item April, May and June (Q2)
\item July, August and September.(Q3)
\item October. November and December (Q4)
\end{itemize}
\parencite{Quarter}
\end{itemize}

\newpage
\subsection{The columns}
\begin{table}[!ht]
\begin{center}
    \begin{tabular}{| l | p{5cm} |}
    \hline
     Column & Description\\ \hline
     Created Date & When a task is put in backlog \\ \hline
     Date From & When a given task is taken from the backlog\\ \hline
     Date to & When a task is done. Done is defined by SI to be ready for release. \\ \hline
    Lead Time & The amount of time elapsed from the date the task was created until the tasks has finished  \\ \hline
   Process Type &States the process used by the team which contains Kanban or Scrum \\ \hline
   Type & The type column is labeled as either bug or feature depending on the type of the task \\ \hline
   Lead time & Number of days used on a tasks \\ \hline
   Lines added & Number of lines added to a feature or bug \\ \hline
   Lines modified & Number of lines modified when working on a feature or bug \\ \hline
   Lines deleted & Number of lines deleted from a bug or feature \\
    \hline
    Team &States the team who has been working on the task.\\ \hline
    \end{tabular}
\caption{Information about the columns from the SI dataset}
\label{IC} %% information columns
\end{center}
\end{table}
\newpage

\subsection {Columns needed for calculating WIP, Lead time, Churn, Bugs Throughput per day}
\begin{table}[!ht]
\begin{center}
    \begin{tabular}{| l | p{5cm} |  p{5cm} |}
    \hline
    Variable &	Description	 & Columns from SI\\ \hline 
     WIP per day & \parbox[t]{5cm}{The number of items in progress on the given day} & Date From and Date To. \\ \hline
     Throughput	& Number of tasks finished on a given day & Date To \\ \hline
     Backlog & Number of items in backlog on a given day & Created Date and Date From\\ \hline
     Churn & Lines added, lines modified and lines deleted added together & Lines added, lines modified and lines deleted \\ \hline
    Bugs & The number of tasks labeled as Bug and not feature & Type \\ \hline
    Lead time & The time used on a task & Lead time \\ \hline
  \end{tabular}
\caption{Information about the columns from the SI dataset}
\label{des} %% desription
\end{center}
\end{table}

\section {WIP-limit per day}
\label{WPD}

%% Table \ref {dataset} show an excerpt from the SI dataset.  The date 2008-10-09 is not in the set so the the program has to create 2008-10-09 and add it to the set in order to calculate WIP, throughput, bugs and lead time per day.  Skrive her at program lager hver dato for set self

\subsection{Step 1: Gather all unique dates into a Arraylist}
\label{sub:stepOne}
First step of WIP measurement is adding every date in the date from column into a Arraylist. Before the date is added, the program checks if the date is already in the Arraylist as shown in line two in listing \ref{lst:Arraylist}. If the date contains in the Arraylist, then the date has already been measured, so the program will discard the date. The Arraylist will contain WIP objects, which will contain information about each date. The information about the object is stated in table \ref{tab:object} %%ref.
\begin{table}[!ht]
\begin{center}
\begin{tabular}{| l | l |}
\hline
Type & Variable name \\ \hline
Date & start \\ \hline
Date & end\\ \hline
String & team\\ \hline
String & processType\\ \hline
int  & WIP\\ \hline
\end{tabular}
\caption{Variables of the WIP objects}
\label{tab:object}
\end{center}
\end{table}
%skrive her en tabell over hvordan objektet ser ut  
 \footnote{Arraylist is a resizable array implementation of a list. The Arraylist class provides function for manipulate the size of the array, check the size of the list and convert the list to an array  \parencite{Arraylist}}
 

Pseudocode step 1:
 \begin{lstlisting}[caption={Gather all unique dates into Arraylist},label={lst:Arraylist}]
for date IN date_from_column:
	if date not contains in Arralist
		WIP = nr_of_date_occurrence(date)
		Arraylist.put(WIP)

int nr_of_date_occurrence(Date date) 
for d IN date_from_column DO
	if d EQUALS date DO
		nr_of_date_occurrence++
return nr_of_date_occurrence
 \end{lstlisting}
\subsection{Step 2: Gather the remaining dates}
 \label{sub:stepTwo}
The data set from SI contains dates from 2008 to 2013, but there are some dates missing which table \ref{dataset} show. In table \ref{dataset} the date 2008-10-09 is missing. In order to generate WIP for each day the program has to create the dates that's not in the set. 
  
In order to create the remaining dates, the program takes the first date and the last date from the Arraylist created in previous section (\ref{sub:stepOne}) as shown in listing \ref{lst:remaining} line one and two. Then the program checks if all the dates between the first date and the last date are in the Arraylist. If the dates are not in the Arraylist, the program will put the date into the Arraylist as show in method on line ten to thirteen.
In order to keep the pseudocode simple, the generateWIP method stated in line twelve was omit. The method creates a new WIP object and returns it.
\begin{minipage}{\textwidth} 
\begin{lstlisting}[caption={Gather the remaining dates},label={lst:remaining}]
WIP first = Arraylist.get(0)//points to the first WIP object in the Arraylist 
WIP last = Arraylist.get(Arraylist.size())//points to the last WIP object in the Arraylist 
Next_date //points to the next date
Next_date = first.getDate() // Next_date assigned before iteration
while Next_date NOT EQUALS last.getDate()
	New_date = Next_Date + 1 //Finds the next date
	AddToArraylist(New_date)
	Next_date = New_date

addToArraylist(Date d)
	if d NOT CONTAINS IN Arraylist
		WIP = generateWIP(d)
		Arraylist.add(WIP) 
 \end{lstlisting}
\end{minipage}
 
\subsection{Step 3 Measure WIP}
The Arraylist from section \ref{sub:stepOne}  and \ref{sub:stepTwo} now contain a WIP object for each date from 2008 to 2013. The reader needs remember the WIP counter of every WIP objects contains the number of tasks started on the corresponding date. This measurement was done in section \ref{sub:stepOne}. In this step the program will loop through the Arraylist, during the iteration each WIP object is extracted from the Arraylist and the new WIP is measured based on how many tasks that has been finished illustrated in the listing \ref{lst:measure}.

\begin{lstlisting}[caption={Meausre WIP},label={lst:measure}]
lastWIP =  0
CurrentWIP = 0
for WIP Object IN Arraylist	
	CurrentWIP = WIP.getWIP() 
	Nr_of_finishedDates = Occurrence_of_date(WIP.getDate())  
	WIP_measured = CurrentWIP - Nr_of_finishedDates + lastWIP
	WIP.setWIP(WIP_measured)
	WIP.setDate(WIP.getDate())
	newArrayList.add(newWIP)
	currentWIP = WIP_measured 

Occurrence_of_date(Date date)
	for d in The Date To column
		if date AFTER d DO
			Nr_of_dates_to_decrement++
return Nr_of_dates_to_decrement 
%% Her mangler jeg � skrive at jeg setter isPicked slik at datoer ikke blir measuret flere ganger, burde jeg ta med det?
 \end{lstlisting}

\section{Example}
\label{sec:Example}
This section will provide a comprehensive example of how the program WIP algorithm works. 
Figure \ref{wip_timeline}  shows tasks id in the y-axis and dates in the x-axis. The green line indicates the duration of the task. The figure visualize how many work in progress (WIP) there are. Such as on the date 2008-10-12, tasks 3, 5 and 6 are in progress, which means the WIP is 3 for 2008-10-12.  

In the example dates from figure \ref{wip_timeline} and from table \ref{wt:2} will be used to illustrate how the algorithm measure WIP
%skrive om tabellen
\begin{figure}[ht!]
\centering
\hspace*{-1in}
\includegraphics[width=21cm,trim=4 4 4 4,clip]{Picture/wip_example.jpg}
\caption{Illustrating the WIP timeline for this example}
\label{wip_timeline}
\end{figure}

\newpage
\begin{table}[!ht]
\begin{center}
    \begin{tabular}{| l | l | p{5cm} |}
    \hline
   Task ID &   Date From  & Date To\\ \hline
     1 & 2008-10-07 & 2008-10-08   \\ \hline
     2 & 2008-10-07 & 2008-10-07   \\ \hline
     3 & 2008-10-09 & 2008-10-16   \\ \hline
     4 & 2008-10-09 & 2008-10-10   \\ \hline
     5 & 2008-10-09 & 2008-11-04   \\ \hline
     6 & 2008-10-10 & 2008-11-05   \\ \hline
     7 & 2008-10-10 & 2008-10-10   \\ \hline
     8 & 2008-10-13 & 2008-10-15   \\ \hline
     9 & 2008-10-13 & 2008-10-13   \\ \hline
    \end{tabular}
\caption{Showing Task ID, Date From and Date to}
\label{wt:2} %%  wip table 2
\end{center}
\end{table}


\subsection{First step}
\label{subsubsec:ft}
First the algorithm will do a look-up on the date 2008-10-07 referred with task ID one and two in table \ref{wt:2} and measure the current WIP counter, which is two, because its two tasks in progress on this date. After this the program will put the date and the current WIP into a Arraylist. 
Now the algorithm will see if any task was done in the last period, since task one and two is the two first tasks to be measured in this example, there's no task finished in the period and there's no current WIP from other tasks, so the current WIP at 2008-10-07 is two illustrated by figure \ref{wip_timeline}. 
\subsection{Second step}
The program will do a look-up on the date 2008-10-09 referred with task ID three, four and five in table \ref{wt:2}  and measure the current WIP for this date, which is three. Then, the program will put the date and the current WIP into a Arraylist.
Next, the program will see that two tasks were done in the period of 2008-10-07 to 2008-10-09. So the program will measure that three new tasks were started at 2008-10-09, two where finished and the WIP from last date measure where two, so the calculation of WIP at 2008-10-09 is 3-2+2, which gives a WIP of three on 2008-10-09 illustrated by figure \ref{wip_timeline}. 

\subsection{Third step}
The program will do a look-up on the date 2008-10-10 referred with task ID six and seven in table \ref{wt:2}  and measure the current WIP for this date, which is two. Next the program will put the date and the current WIP into a Arraylist.
Last the program will see that no task where done in the period 2008-10-09 to 2008-10-10 and the currently WIP is three, so this gives a new WIP of 5 illustrated by figure \ref{wip_timeline}. 

\subsection{Fourth step}
The program will do a look-up on the date 2008-10-13 referred with task ID eight and nine in table \ref{wt:2}  and measure the current WIP for this date, which is two. Then the program will put the date and the current WIP into a Arraylist. 
Next the program will look at the period between 2008-10-10 and 2008-10-13, and see that task four and seven was done in this period. The WIP at 2008-10-13 will be $2-2+5 = 5$ as illustrated by figure \ref{wip_timeline}. 
As illustrated by the example, WIPs are not decrement until the finished date is passed, even though the task is done on the date, it's also worked on the same date, therefore the WIP is decrement after each date is passed.

\section {Bug}
\label{Bug}
Each task in the data set is label as either feature or bug as shown in figure \ref{bugsT}. When the program reads in the data file, each task label as bug are saved in a data structure. The code is shown in listing \ref{findBug}.

\begin{table}[ht]
\begin{center}
    \begin{tabular}{| l | l | p{5cm} |}
    \hline
    Task ID & Type \\ \hline
57970 &	Bug\\ \hline
57971&	Bug\\ \hline
57972&	Bug\\ \hline
57973&	Bug\\ \hline
57974&	Feature\\ \hline
57975&	Feature\\ \hline
57976&	Feature\\ \hline
57977&	Feature\\ \hline
57978&	Feature\\ \hline
    \end{tabular}
\caption{Example of how tasks are labeled}
\label{bugsT} %%  bugs table
\end{center}
\end{table}

\begin{minipage}{\textwidth}
\begin{lstlisting}[caption=Pseudocode example of how bugs are found, label=findBug]
void findBug()
	while inputFile != EOF // End Of File
		newLine = readLine()
		if newLine.Type EQUALS 'Bug'
			B = New Bug()
			B.startDate = newLine.startDate
			B.process = newLine.process
			B.team = newLine.team 
			AddNewBug(B)
\end{lstlisting}
\end{minipage}
\subsection{Add Bug}
\label{addBugS}
When adding a new bug, each bug is gathered into a data structure based on which team the bug belongs to, as the tests in lines 2, 7, 12 and 17 of the listing \ref{addBug} shows. After the right team is found the program tries to add the bug to the corresponding data structure as illustrated in lines 3, 8, 13 and 18. If the date of newly arrived bug already contains in the data structure, a counter representing the date is incremented and the new bug is discarded as shown the method dateExists in lines 26 to 32. 

Since the program knows which team the new bug belongs to (after the checks on line 3, 8, 13 and 18) a counter can represent each bug. In our analyze the only important for us is to know how many bugs there are on a given date and which team it belongs to.  
\newpage
\begin{lstlisting}[caption=Pseudocode example of how bugs are added, label=addBug]
void addBug(Bug b)
	if b.team EQUALS "360"
		if dateExists(b.date, 360.dataStructure) EQUALS false
			// if date don't exists, then add the bug
			360.dataStructure.add(b)
			
	if b.team EQUALS "Neon"
		if dateExists(b.date, Neon.dataStructure) EQUALS false
			// if date don't exists, then add the bug
			Neon.dataStructure.add(b)
			
	if b.team EQUALS "Frontend"
		if dateExists(b.date, Frontend.dataStructure) EQUALS false 
			// if date don't exists, then add the bug
			Frontend.dataStructure.add(b)
			
	if b.team EQUALS "Krypton"
		if dateExists(b.date, Krypton.dataStructure) EQUALS false
			// if date don't exists, then add the bug
			Krypton.dataStructure.add(b)
		
void dateExists(Date d, DataStructure structure)
	for Bug b in structure
		if b.date EQUALS d
			b.counter++
			return true
	
return false	
 \end{lstlisting}
 
\section{Throughput}
 \label{TP}
Finding the throughput per day is quite similar to how bugs are found (described in section \ref{Bug}).  When reading in the data set a new throughput object is created for each line in the data set.  Then all throughput objects are sorted based on team association. When all throughput objects are sorted, the program measure throughput. The throughput measurement is similar to the dateExists method (lines 26 to 32) in the listing \ref{addBug} stated in section \ref{addBugS}. 

\newpage
The dateExists method starts of with a test, the same test is done for bugs. If the date of the throughput object is in the data structure the corresponding counter is incremented.  If the date is not in the data structure, the new throughput object is added to the data structure. An excerpt of the code is listed in \ref{throughputCode} 

\begin{lstlisting}[caption=Pseudocode example of how throughput is measured, label=throughputCode]

dateExists(Throughput tp d, DataStructure structure)
	for Throughput t in structure
		if t.date EQUALS tp.date
			t.counter++
			return
			
structure.add(tp);
\end{lstlisting}


\section{Churn}
\label{churn}
As stated in section \ref{sec:Churn} in order to take churn into account one need to know it's good surrogates. SI has gathered churn with help of Microsoft's Team Foundation Server (TFS). The TFS system automatically records data such as churn and lead time. Based on TFS one can know that churn for SI a is good surrogate.

To measure churn the data set from SI contains three columns as shown in table \ref{Churn}. To complete the measure the three columns are multiplied.  For task id one, the churn is 2028 ($352+307+1369 = 2028$). The task with id six has zero churn. In the data set there exists tasks with zero churn, these tasks don't need code in order to be completed such tasks needs support to be finished.

%skal jeg skrive pseucode code til denne? 

\begin{table}[ht]
\begin{center}
    \begin{tabular}{| l | l | l | l |}
    \hline
    Task id & Lines added & Lines modified  & Lines deleted \\ \hline
1&352&307&1369\\ \hline
2&314 & 31 & 15 \\ \hline
3&314&31 & 15\\ \hline
4&62&327&153 \\ \hline
5&21&3&0 \\ \hline
6&0&0&0 \\ \hline
\end{tabular}
\caption{How churn is presented in the excel document}
\label{Churn} %%  bugs table
\end{center}
\end{table}

\section{SPSS}
\subsection{Introduction}
After the program has analyzed the data from SI,  SPSS will be used on the data derived from from the program. SPSS will help to answer the research question stated in chapter \ref{chap:RQ} with help of these three statistics method: Correlation, Graph's  and case summaries. The case summaries will be used on:
\begin{itemize}[noitemsep,topsep=0pt,parsep=0pt,partopsep=0pt]
\item WIP
\item Throughput
\item Lead time
\item Churn
\item Bugs
\end{itemize}
\subsection{SPSS}
"IBM� SPSS� Statistics is a comprehensive system for analyzing data. SPSS Statistics can take data from almost any type of file and use them to generate tabulated reports, charts and plots of distributions and trends, descriptive statistics, and complex statistical analyses." \parencite{IBM}

From SPSS the function stated in figure \ref{SPSSfunctions} will be used to analyze data.
\begin{figure}[!ht]
\begin{itemize}
\item Case summaries
\item Line Graph
\item Correlation
\caption{SPSS functions used to analyze data in this thesis}
\label{SPSSfunctions}
\end{itemize}
\end{figure}
\newpage
\subsubsection{Case summaries}
The case summarize function is used on WIP, throughput, lead time, bugs and churn to calculate in variable in figure \ref{DS} per quarter:
\begin{figure}[!ht]
\begin{itemize}[noitemsep,topsep=0pt,parsep=0pt,partopsep=0pt]
\item Mean
\item Median
\item Standard Derivation
\item Number (N)
\item Max
\item Min
\caption{Descriptive statistic}
\label{DS}
\end{itemize}
\end{figure}


%skrive her om descriptive statistics go

\subsubsection{Line Graph}
The line graph function is used to line graph with both one and two y-axis. 

\subsubsection{Correlation}
The correlation function is used to look at the relationship between variables. In this thesis the correlation function will help determine if there are any relationship between WIP, throughput, bugs, lead time and churn.  In order to determine the relationship between variables the variables correlation goes from -1 to 1.
\begin{itemize}
\item If the correlation is between $ \mypm 0.4 $ it's a low correlation
\item If the correlation is between $ \mypm 0.5-0.8$ it's a medium correlation
\item If the correlation is between $ \mypm 0.8 $ it's a strong correlation
\item If the correlation is positive it means that the variables are decreasing or increasing at the same time. 
\item If the correlation is negative it means that the variables are going in the opposite direction. 
\item If the correlation is low, it means that there is no detective pattern between the variable and there is no relationship exists between the variable.
\item If the correlation is high, it means that the exists a pattern between the variables. 
\end{itemize}



%%\part{The project} 
\chapter{Results}                     %% ... or ??

 
\section{Introduction}
In this chapter 


\section {Team 1}
\newpage
\subsection{WIP correlation}
\label{sec:WIPC}
As shown in table \ref{corr:1} bugs, throughput, churn, churn bugs and throughput bugs have a significant correlation with WIP. In table \ref{DS:1} is descriptive statistic for WIP shown.  As shown by the N column all quarters besides 2010-3 give reasonable data since in each quarter has 90 days give or take. 

% hadde SI or 360 WIP-limit?
\begin{table}[!htbp] 
 \centering 
 \begin{tabular}{|l|l|l|l|l|l|} 
\hline 
Correlation  & Bugs & Throughput & Churn & Churn Bugs & Throughput bugs \\ \hline 
	WIP & .753* & .637** & .738* & .708* & .800** \\ \hline 
\end{tabular} 
 \caption{Team one - Correlation - 	WIP} 
 \label{corr:1}
    \centerline {* Correlation is significant at the 0.05 level (2-tailed).}
      \centerline{  ** Correlation is significant at the 0.01 level (2-tailed).}
 \end{table}  
 
 \begin{table}[!htbp]
  \centering
  \scalebox{1}{
   \begin{tabular}{ | l | l | l | l | l | l | l | l | }
\hline
	TeamName & Quarter & N & Mean & Median & Std. Deviation & Maximum & Minimum \\ \hline
	360 & 2010-3 & 39 & 5.82 & 5 & 2.55 & 15 & 1 \\ \hline
	 & 2010-4 & 92 & 1.7 & 2 & 0.70 & 4 & 1 \\ \hline
	 & 2011-1 & 90 & 4.37 & 2 & 6.87 & 31 & 1 \\ \hline
	 & 2011-2 & 91 & 14.25 & 5 & 14.497 & 52 & 3 \\ \hline
	 & 2011-3 & 92 & 2.4 & 3 & 0.97 & 4 & 1 \\ \hline
	 & 2011-4 & 92 & 14.26 & 4 & 22.718 & 97 & 1 \\ \hline
	 & 2012-1 & 91 & 13.35 & 5 & 16.62& 67 & 4 \\ \hline
	 & 2012-2 & 91 & 15.35 & 3 & 27.145 & 94 & 2 \\ \hline
	 & 2012-3 & 92 & 16.97 & 13.5 & 8.719 & 52 & 5 \\ \hline
	 & 2012-4 & 90 & 9.07 & 2 & 17.073 & 74 & 1 \\ \hline
	 & Total & 860 & 9.99 & 4 & 16.04 & 97 & 1 \\ \hline
	\end{tabular}
       }
      \caption{Descriptive statistics for WIP - Team one}
  \label{DS:1}%
\end{table}% 
\newpage

\subsubsection{WIP and throughput correlation}
Both throughput and throughput bugs have a significant correlation relationship with WIP. This is inconsistent with Kanban philosophy and what the research says about WIP-limit.  All the research and the principles of Kanban suggest lowering WIP in order to get high throughput. In team one's case the higher the WIP the higher the throughput.  

In table \ref{DS:1} the mean column shows indication that team one could have experimented with WIP-limit the first five quarters. But in the last quarter, the WIP-limit decreased.  In the mean column in table \ref{DS:T:1}  if one take a look at the quarter 2012-4, the throughput are second highest, the same quarter as WIP decreased.  This link could have several reasons. One could be that they experimented with WIP and found out after the quarter 2012-3 that they needed to lower the WIP in order to increase the t hroughput, which it did.  Or maybe team one didn't care about setting a WIP-limit as stated in section \ref{WIPsec} by Shinkle. %ha med ref her til artikkelen?

The throughput bugs variable and throughput variable both have a significant correlation with WIP, which make sense since throughput bugs are a subset of throughput. But throughput feature on the other hand has a -0.441 correlation with WIP. % burde jeg her legge inn -0.441 inn I en tabell? I samme correlationstabellen?
As shown in table \ref{NoT:1}. The correlation number -0.441 is not a significant correlation, but it's on the opposite site level of throughput and throughput feature although it's also a subset of throughput.  This also goes in the literatures favor, because when WIP is low, more throughput feature is produced. Maybe this could have something to do with the relationship bugs and feature? Features are often more challenging and more fun to work with than bugs.
One could also argue that maybe it's was significant fewer tasks labeled as bug than feature or vice versa, but as shown in table \ref{NoT:1} that's not the case. 

%2012-3 releases? Det kan v�re en sammenheng med at median er s� h�y der
   \begin{table}[!htbp]
 \begin{tabular}{ | l | l | l | l | l | l | l | l | }
\hline
	Team & Quarter & N & Mean & Median & Std. Deviation & Maximum & Minimum \\ \hline
	360 & 2010-3 & 5 & 2.6 & 2 & 1.34 & 4 & 1 \\ \hline
	 & 2011-1 & 7 & 5.71 & 6 & 4.03 & 13 & 1 \\ \hline
	 & 2011-2 & 26 & 3.38 & 3 & 1.96 & 8 & 1 \\ \hline
	 & 2011-3 & 1 & 1 & 1 & . & 1 & 1 \\ \hline
	 & 2011-4 & 15 & 6.13 & 4 & 4.984 & 18 & 2 \\ \hline
	 & 2012-1 & 16 & 6.56 & 6.5 & 3.59 & 14 & 1 \\ \hline
	 & 2012-2 & 13 & 10.15 & 10 & 5.742 & 23 & 1 \\ \hline
	 & 2012-3 & 22 & 3.41 & 2 & 3.33 & 12 & 1 \\ \hline
	 & 2012-4 & 14 & 8.14 & 7 & 6.5 & 21 & 1 \\ \hline
	 & Total & 119 & 5.55 & 4 & 4.694 & 23 & 1 \\ \hline
	 \end{tabular}
	 \caption{Descriptive statistic for throughput - team one }
	 \label{DS:T:1}
	 \end{table}%
	 
\begin{table}[!htbp]
\begin{tabular}{ | l | l | l | l | l | l | l | }
\hline
	Quarter & N & Mean & Median & Std. Deviation & Maximum & Minimum \\ \hline
	2010-3 & 14.00 & 3.07 & 2.00 & 3.47 & 14.00 & 1.00\\ \hline
	2010-4 & 71.00 & 2.73 & 2.00 & 2.05 & 11.00 & 1.00\\ \hline
	2011-1 & 67.00 & 2.55 & 2.00 & 1.96 & 13.00 & 1.00\\ \hline
	2011-2 & 38.00 & 1.97 & 2.00 & 1.03 & 4.00 & 1.00\\ \hline
	2011-3 & 43.00 & 1.65 & 1.00 & 0.92 & 4.00 & 1.00\\ \hline
	2011-4 & 53.00 & 1.83 & 1.00 & 1.31 & 6.00 & 1.00\\ \hline
	2012-1 & 40.00 & 1.90 & 1.00 & 1.46 & 7.00 & 1.00\\ \hline
	2012-2 & 27.00 & 2.07 & 2.00 & 1.36 & 6.00 & 1.00\\ \hline
	2012-3 & 40.00 & 2.70 & 2.00 & 2.43 & 13.00 & 1.00\\ \hline
	2012-4 & 3.00 & 1.00 & 1.00 & 0 & 1.00 & 1.00\\ \hline
	Total & 396.00 & 2.26 & 2.00 & 1.82 & 14.00 & 1.00\\ \hline
	 \end{tabular}
\caption{Descriptive statistic for throughput feature - team one }%
\end{table}	 

\begin{table}[!htbp]	 
\begin{tabular}{ | l | l | l | l | l | l | l | l | }
\hline
	TeamName & Quarter & N & Mean & Median & Std. Deviation & Maximum & Minimum \\ \hline
	360 & 2010-3 & 9 & 1.11 & 1 & 0.33& 2 & 1 \\ \hline
	 & 2010-4 & 3 & 1 & 1 & 0 & 1 & 1 \\ \hline
	 & 2011-2 & 26 & 5 & 3 & 4.45 & 17 & 1 \\ \hline
	 & 2011-3 & 1 & 1 & 1 & . & 1 & 1 \\ \hline
	 & 2011-4 & 24 & 6.79 & 4 & 6.35 & 20 & 1 \\ \hline
	 & 2012-1 & 18 & 4.83 & 4 & 3.20 & 11 & 1 \\ \hline
	 & 2012-2 & 19 & 7.84 & 9 & 5.843 & 17 & 1 \\ \hline
	 & 2012-3 & 6 & 2.67 & 1.5 & 2.25 & 6 & 1 \\ \hline
	 & 2012-4 & 11 & 3.18 & 3 & 1.77 & 7 & 1 \\ \hline
	 & Total & 117 & 5.08 & 3 & 4.88 & 20 & 1 \\ \hline
	 \end{tabular}
\caption{Descriptive statistic for throughput bugs - team one }%
\end{table}  

\begin{table}[!htbp]
\centering
\begin{tabular}{ | l | l | }
\hline
	Number of features & Number of bugs \\ \hline
	660 & 594 \\ \hline
\end{tabular}
   \caption{Number of throughput divided into bugs and feature - Team one}
  \label{NoT:1}%Nr of throughput 
\end{table}% 
 
\newpage

\subsubsection{WIP and bugs correlation}
Only looking at WIP and throughput correlation without taking other factores into account may give bias.

Bugs and churn have also have a significant correlation with WIP.  A positive bug correlation could help favor what said about WIP and throughput in literature. Since both WIP and throughput increases and decreases at the same time, the same goes for bug. This could mean that when WIP is high, and throughput is high team one also has a lot of bugs in their code. This means that they don't do their job well, and this could be because they have a lot of task switching due to high WIP-limit. As stated in table \ref{PB:1} one can see that almost each bug that was reported was finished in the same quarter. The outcast could be the bugs reported just before the quarter end. Worth noticing in quarter 2011-1 there was no bugs recorded. This conclude the statement that the WIP-limit is set too high. 0.467
%2012-3 releases? Det kan v�re en sammenheng med at median er s� h�y der

\begin{table}[!htbp]
\centering
\begin{tabular}{ | l |l | }
\hline
	Quarter & Precent of bugs finished in the same quarter as reported   \\ \hline
	2010-3 & 100\%   \\ \hline
	2010-4 & 100\%   \\ \hline
	2011-1 & 0\%    \\ \hline
	2011-2 & 97.74\%  \\ \hline
	2011-3 & 20\%   \\ \hline
	2011-4 & 97.5\%  \\ \hline
	2012-1 & 96.51\%  \\ \hline
	2012-2 & 98.63\%   \\ \hline
	2012-3 & 82.35\%  \\ \hline
	2012-4 & 100\%    \\ \hline
\end{tabular}
   \caption{The percent number of bugs finished in the same quarter as reported - Team one}
  \label{PB:1}%Precent bugs team 1
\end{table}% 
\newpage

\newpage

\subsubsection{WIP and churn correlation}
Churn and WIP have a positive correlation. This means that when team one has more tasks in progress the tasks requires more coding. This is odd and could be a coincidence. Churn and churn bugs have also here as with throughput a close correlation with WIP. But churn feature has a correlation of 0.467 %legge inn I tabell? komme med et poeng her? 
 
As shown in table \ref{DS:CF:1}  and the column 'maximum' the biggest feature task requires almost 60 times more code editing than the biggest bug task as showed in table \ref{DS:CB:1}. In the same table the column std. deviation shows way more proliferation for churn feature. This indicates that the numbers in churn feature are way more spread, which again indicates that it's random of how big the tasks are. In the churn bugs the std. deviation is much lower than in the feature table. This could explain why churn feature not has a positive correlation as churn and churn bugs.  Both median and mean favors that churn feature requires more work than churn bugs



% This could due to releaser sjekke ut dette.  Hvis det har noe med releaser s� er det slik at n�r de er n�rmere en release s� blir oppgavene st�rre siden de har mer � gj�re for releasen kommer. Er det slik at SI da har satt seg I et m�l om  � levere s� og s� mye slik som de gjorde I Scrum?

\begin{table}[!htbp]
\centering
\begin{tabular}{ | l | l | l | l | l | l | l | }
\hline
	Quarter & N & Mean & Median & Std. Deviation & Maximum & Minimum \\ \hline
	2010-3 & 5 & 434 & 106 & 808.67 & 1879 & 30 \\ \hline
	2011-1 & 7 & 0 & 0 & 0 & 0 & 0 \\ \hline
	2011-2 & 26 & 1199.54 & 139 & 4695.424 & 24139 & 0 \\ \hline
	2011-3 & 1 & 0 & 0 & . & 0 & 0 \\ \hline
	2011-4 & 15 & 516.4 & 148 & 1063.92 & 3946 & 0 \\ \hline
	2012-1 & 16 & 5775.25 & 410 & 20692.09 & 83317 & 0 \\ \hline
	2012-2 & 14 & 7569.14 & 522.5 & 24655.70 & 93108 & 0 \\ \hline
	2012-3 & 23 & 6.74 & 0 & 25.44 & 118 & 0 \\ \hline
	2012-4 & 16 & 408 & 47 & 1056.23 & 4275 & 0 \\ \hline
	Total & 123 & 2001.29 & 65 & 11380.06 & 93108 & 0 \\ \hline
\end{tabular}
   \caption{Descriptive Statistic for churn feature - Team one}
  \label{DS:CF:1}% descriptive statistic churn feature team 1
\end{table}%

\begin{table}[!htbp]
\centering
\begin{tabular}{ | l | l | l | l | l | l | l | }
\hline
	Quarter & N & Mean & Median & Std. Deviation & Maximum & Minimum \\ \hline
	2010-3 & 7 & 57.14 & 22 & 73.91 & 193 & 4 \\ \hline
	2010-4 & 2 & 30 & 30 & 41.012 & 59 & 1 \\ \hline
	2011-2 & 21 & 386.33 & 177 & 431.94& 1604 & 2 \\ \hline
	2011-3 & 1 & 2 & 2 & . & 2 & 2 \\ \hline
	2011-4 & 19 & 306.89 & 148 & 437.13 & 1557 & 12 \\ \hline
	2012-1 & 16 & 312.81 & 160 & 307.73& 1035 & 4 \\ \hline
	2012-2 & 13 & 296.92 & 169 & 292.29 & 1019 & 18 \\ \hline
	2012-3 & 7 & 37.71 & 9 & 56.494 & 126 & 0 \\ \hline
	2012-4 & 8 & 118.75 & 84 & 103.09 & 292 & 3 \\ \hline
	Total & 94 & 260.48 & 135 & 346.25 & 1604 & 0 \\ \hline
\end{tabular}
 \caption{Descriptive Statistic for churn bugs - Team one}
  \label{DS:CB:1}% descriptive statistic churn bugs team 1
\end{table}%
 \newpage 

\subsection{Throughput}
In table \ref{TP:C:1} is correlation for throughput for team one stated. Throughput has a positive correlation to bugs, WIP, churn, churn bugs, churn feature, throughput bugs and churn average. In table \ref{DS:T:1:2} is descriptive statistic for throughput shown. The range from 2011-2 to 2012-4 is the most valuable quarters based on the number of tasks finished if one excludes the quarter 2011-3.
\begin{table}[!htbp] 
 \centering
 \hspace*{-0.5in} 
 \begin{tabular}{|l|l|l|l|l|l|l|l|} 
\hline 
Correlation  & Bugs & WIP & Churn & Churn Bugs & Churn feature & Throughput bugs & Churn Average \\ \hline 
	Throughput & .945** & .816** & .962** & .776** & .736* & .940** & .638* \\ \hline
\end{tabular} 
 \caption{Team one - Correlation - 	Throughput} 
 \label{TP:C:1}
     \centerline {* Correlation is significant at the 0.05 level (2-tailed).}
      \centerline{  ** Correlation is significant at the 0.01 level (2-tailed).}
 \end{table} 
 
    \begin{table}[!htbp]
 \begin{tabular}{ | l | l | l | l | l | l | l | }
\hline
	 Quarter & N & Mean & Median & Std. Deviation & Maximum & Minimum \\ \hline
	 2010-3 & 5 & 2.6 & 2 & 1.34 & 4 & 1 \\ \hline
	 2011-1 & 7 & 5.71 & 6 & 4.03 & 13 & 1 \\ \hline
	 2011-2 & 26 & 3.38 & 3 & 1.96 & 8 & 1 \\ \hline
	 2011-3 & 1 & 1 & 1 & . & 1 & 1 \\ \hline
	 2011-4 & 15 & 6.13 & 4 & 4.984 & 18 & 2 \\ \hline
	 2012-1 & 16 & 6.56 & 6.5 & 3.59 & 14 & 1 \\ \hline
	 2012-2 & 13 & 10.15 & 10 & 5.742 & 23 & 1 \\ \hline
	 2012-3 & 22 & 3.41 & 2 & 3.33 & 12 & 1 \\ \hline
	 2012-4 & 14 & 8.14 & 7 & 6.5 & 21 & 1 \\ \hline
	 Total & 119 & 5.55 & 4 & 4.694 & 23 & 1 \\ \hline
	 \end{tabular}
	 \caption{Descriptive statistic for throughput - team one }
	 \label{DS:T:1:2}
	 \end{table}%
	 
\subsubsection{Throughput and bugs correlation}
The correlation between throughput and bugs is 0.945, which is significantly high. 
The quarters from 2011-2 to 2012-4 minus 2011-3 is also the most valid in bugs. The correlation between bugs and throughput means that when team one has high throughput of tasks they also produce more bugs. This correlation can be used to favor the literature of WIP as stated in section\ref{sec:WIPC}. Since team one produces more bugs when they finished feature or bugs could be an indicator that they may have a too high WIP-limit so they cannot focus enough on each task.

Throughput and bugs also have the same relationship as WIP and bugs, because throughput and throughput bugs has a significant correlation to both, but throughput feature not. Throughput feature have correlation of -0.340 compared to -0.441 in WIP. What this means is that when team one has high throughput of tasks, they mostly produces tasks labeled as bug.  This could be because the feature tasks are more comprehensive than the tasks labeled as bug so the lead time is longer for feature tasks than bug tasks. Or it could due the reason that when there is an upcoming release, all the bugs are pushed through the Kanban system. 

In table \ref{cfb:1} the hypotheses that feature tasks is more comprehensive is verified.  Four out of the six valid quarters churn feature is significant higher.  In table \ref{lfb:1} one can see that throughput bug's lead time is higher than throughput feature in four out of six quarters. The lead time cannot be counted, since bugs usually have longer lead time than feature, because feature usually has higher priority than bugs.   
  
	
%Both of these hypotheses would contradict the principle of WIP or Agile. %for det hvis de pusher bugs gjennom systemet vil WIP-limit stige. Eller hvis tasks er forskjellige st�rrelser s� er det umulig � male TP, som da igjen gj�r det vanskelig � male WIP. 

%Jeg m� sjekke leadtimene og churn for tp feature og tp bugs for 360
%Det andre statementet m� jeg sjekke n�r jeg f�r releaser av Dag

\begin{table}[!htbp] 
 \centering
\begin{tabular}{ | l | l | l | }
\hline
	Quarter & Churn feature & Churn bug \\ \hline
	2010-3 & 0 & 29.3 \\ \hline
	2010-4 & 0 & 20 \\ \hline
	2011-1 & 0 & 0 \\ \hline
	2011-2 & 1199.54 & 59.14 \\ \hline
	2011-3 & 0 & 2 \\ \hline
	2011-4 & 516.4 & 76.66 \\ \hline
	2012-1 & 5775.25 & 1009.9 \\ \hline
	2012-2 & 7569.14 & 707.46 \\ \hline
	2012-3 & 6.74 & 15.68 \\ \hline
	2012-4 & 408 & 168.28 \\ \hline
\end{tabular}
\caption{team one - Average churn for feature and bugs - Bugs }%
\label{cfb:1} %churn feature bugs
\end{table}

\begin{table}[!htbp] 
 \centering
\begin{tabular}{ | l | l | l | }
\hline
	Quarter & Throughput feature & Throughput bug \\ \hline
	2010-3 & 23 & 51.33 \\ \hline
	2011-1 & 38.85 & 6 \\ \hline
	2011-2 & 19.73 & 25.46 \\ \hline
	2011-3 & 1 & 5 \\ \hline
	2011-4 & 30.6 & 36.38 \\ \hline
	2012-1 & 20.37 & 45.56 \\ \hline
	2012-2 & 35.30& 55.89 \\ \hline
	2012-3 & 30.72 & 17.67 \\ \hline
	2012-4 & 30.57& 8.82 \\ \hline
	Total & 27.30 & 35.09 \\ \hline
\end{tabular}
\caption{Lead time for throughput feature and bugs per quarter}
\label{lfb:1} %lead time feature bugs
\end{table}
\newpage 
\subsubsection{Throughput and churn correlation}
Churn, churn bugs, churn feature and the average of churn all have a positive correlation with throughput which mean that when throughput is high, the tasks requires more coding. From table \ref{dsc:1} in the number column one can see that in each quarter there is a lot of tasks. Although from the median column one can see that in 6 of the valid quarters, three of them have a median of 0, which gives and indicator that in these three quarter there is probably minority of tasks which has high churn.  So the correlation between throughput and churn may not be valid.

\begin{table}[!htbp] 
\begin{tabular}{ | l | l | l | l | l | l | l | }
\hline
Quarter & N & Mean & Median & Std. Deviation & Maximum & Minimum \\ \hline
	2010-3 & 11.00 & 17.00 & 11.00 & 18.94 & 60.00 & 0\\ \hline
	2011-1 & 73.00 & 0 & 0 & 0 & 0 & 0\\ \hline
	2011-2 & 136.00 & 14.40 & 3.00 & 21.51 & 83.00 & 0\\ \hline
	2011-3 & 48.00 & 1.67 & 0 & 7.71 & 50.00 & 0\\ \hline
	2011-4 & 156.00 & 14.81 & 6.50 & 19.91 & 86.00 & 0\\ \hline
	2012-1 & 157.00 & 9.47 & 0 & 19.55 & 86.00 & 0\\ \hline
	2012-2 & 238.00 & 10.11 & 0 & 18.50 & 83.00 & 0\\ \hline
	2012-3 & 53.00 & 3.02 & 0 & 9.62 & 53.00 & 0\\ \hline
	2012-4 & 26.00 & 21.42 & 14.50 & 20.55 & 67.00 & 1.00\\ \hline
	Total & 902.00 & 10.14 & 0 & 18.55 & 86.00 & 0\\ \hline
\end{tabular}
\caption{Descriptive statistic for churn - team one }
\label{dsc:1} %
\end{table}


\subsection {Bugs}
\begin{table}[!htbp] 
 \centering 
 \begin{tabular}{|l|l|l|l|l|l|l|} 
\hline 
Correlation  & Throughput & WIP & Churn & Churn Bugs & Churn feature & TP bugs \\ \hline 
	Bugs & .945** & .753* & .992** & .813** & .730* & .984** \\ \hline 

\end{tabular} 
\caption{Team two - Correlation - Throughput }%
   \centerline {* Correlation is significant at the 0.05 level (2-tailed).}
      \centerline{  ** Correlation is significant at the 0.01 level (2-tailed).}
\end{table}  

 \begin{table}[!htbp]
\begin{tabular}{ | l | l | l | l | l | l | l | l | }
\hline
	TeamName & Quarter & N & Mean & Median & Std. Deviation & Maximum & Minimum \\ \hline
	360 & 2010-3 & 7 & 1.29 & 1 & 0.48 & 2 & 1 \\ \hline
	 & 2010-4 & 4 & 1 & 1 & 0 & 1 & 1 \\ \hline
	 & 2011-2 & 32 & 4.16 & 3.5 & 3.63 & 14 & 1 \\ \hline
	 & 2011-3 & 5 & 1 & 1 & 0 & 1 & 1 \\ \hline
	 & 2011-4 & 26 & 6.15 & 4 & 5.80 & 22 & 1 \\ \hline
	 & 2012-1 & 21 & 4.09 & 4 & 2.7 & 11 & 1 \\ \hline
	 & 2012-2 & 17 & 8.59 & 9 & 5.59& 19 & 1 \\ \hline
	 & 2012-3 & 7 & 2.43 & 2 & 1.512 & 5 & 1 \\ \hline
	 & 2012-4 & 11 & 3 & 3 & 2 & 6 & 1 \\ \hline
	 & Total & 131 & 4.53 & 3 & 4.44 t & 22 & 1 \\ \hline
	 \end{tabular}
\caption{360 - Descriptive statistic - Bugs }%

\end{table}

It's hard to find any correlation between lead time, WIP and throughput. In quarter three in 2011, lead time, WIP and throughput dropped to their lowest point. This was due to the summer vacation. In quarter two of 2011 and two of 2012 bugs dropped but both Throughput and WIP increased.  
\newpage


\section{Team 2}
\subsection{WIP}
For team two there is no significant correlation between WIP or any of the variables that have been measured. In table \ref{dsw:2} one can see that every quarter beside 2010 is a valid quarter based on the number of dates in quarter. An excerpt of the correlation data is shown in table \ref{wc:2}.  As in section \ref{sec:WIPC} for team one, throughput is fairly high, not significant high as for team one, but fairly high. For team two the correlation between bugs and WIP is quite low, so the conclusion stated in section \ref{sec:WIPC}  may not suit team two. 

In table \ref{tc:2} the correlation for throughput for team two is shown.  Here one can see that throughput and bugs have a significant correlation. That means when WIP increases, throughput increases and when throughput increases the number of bugs reported increases. Which could mean that the WIP-limit is to high.


\begin{table}[!htbp] 
 \centering 
\begin{tabular}{|l|l|l|l|l|} \hline
&Throughput & Bugs & Churn  & Leadtime  \\ 
\hline
	WIP &    0.341 & 0.278  & -0.446 & 0.096 \\ \hline
 \end{tabular}
  \caption{Team two - Correlation - WIP} 
  \label{wc:2}
  \centerline {* Correlation is significant at the 0.05 level (2-tailed).}
      \centerline{  ** Correlation is significant at the 0.01 level (2-tailed).}
 \end{table} 
 
 
  \begin{table}[!htbp]
  \begin{tabular}{ | l | l | l | l | l | l | l | l | }
\hline
TeamName & Quarter & N & Mean & Median & Std. Deviation & Maximum & Minimum \\ \hline
Frontend & 2010-3 & 24 & 9.28 & 10 & 6.43 & 23 & 1 \\ \hline
	 & 2010-4 & 92 & 13.92 & 13 & 3.931 & 25 & 5 \\ \hline
	 & 2011-1 & 90 & 15.3 & 15.5 & 3.94 & 23 & 7 \\ \hline
	 & 2011-2 & 91 & 23.51 & 24 & 4.19 & 37 & 13 \\ \hline
	 & 2011-3 & 92 & 20.13 & 20 & 5.43 & 32 & 9 \\ \hline
	 & 2011-4 & 92 & 21.33 & 21 & 6.92 & 34 & 7 \\ \hline
	 & 2012-1 & 91 & 22.91 & 22 & 6.63 & 40 & 11 \\ \hline
	 & 2012-2 & 91 & 21.93 & 21 & 3.36 & 32 & 17 \\ \hline
	 & 2012-3 & 92 & 25.93 & 27 & 4.92 & 36 & 19 \\ \hline
	 & 2012-4 & 67 & 19.18 & 15 & 9.66 & 41 & 9 \\ \hline
	 & Total & 822 & 20.18 & 20 & 6.93 & 41 & 1 \\ \hline
	 \end{tabular}
  	  \caption{Descriptive statistic - WIP - Team two }%
	  \label{dsw:2}
\end{table}	



\begin{table}[!htbp] 
 \centering 
 \begin{tabular}{|l|l|l|l|} 
\hline 
Correlation  & Bugs & Throughput feature & Bugs finished in the same quarter\\ \hline  %Nr of bugs finished in the same quarter precent
	Throughput & .885** & .669* & .640* \\ \hline 

\end{tabular} 
 \caption{Team one - Correlation - Throughput} 
 \label{tc:2}
  \centerline {* Correlation is significant at the 0.05 level (2-tailed).}
      \centerline{  ** Correlation is significant at the 0.01 level (2-tailed).}
 \end{table} 
 \newpage
\subsection {Lead time}
\begin{table}[!htbp] 
 \centering 
 \begin{tabular}{|l|l|l|l|l|} 
\hline 
Correlation  & Churn Bugs & Churn union & Precent bugs fininshed & Churn Average \\ \hline 
	Leadtime & -.771** & .704* & -.697* & .766** \\ \hline
\end{tabular} 
 \caption{Team one - Correlation - 	Leadtime\_union} 
 \end{table}   
 
    \begin{table}[!htbp]
   \centering
 \begin{tabular}{ | l | l | l | l | l | l | l | }
\hline
	Quarter & N & Mean & Median & Std. Deviation & Maximum & Minimum \\ \hline
	2010-3 & 14.00 & 3.07 & 2.00 & 3.47 & 14.00 & 1.00\\ \hline
	2010-4 & 71.00 & 2.73 & 2.00 & 2.05 & 11.00 & 1.00\\ \hline
	2011-1 & 67.00 & 2.55 & 2.00 & 1.96 & 13.00 & 1.00\\ \hline
	2011-2 & 38.00 & 1.97 & 2.00 & 1.03 & 4.00 & 1.00\\ \hline
	2011-3 & 43.00 & 1.65 & 1.00 & 0.92 & 4.00 & 1.00\\ \hline
	2011-4 & 53.00 & 1.83 & 1.00 & 1.31 & 6.00 & 1.00\\ \hline
	2012-1 & 40.00 & 1.90 & 1.00 & 1.46 & 7.00 & 1.00\\ \hline
	2012-2 & 27.00 & 2.07 & 2.00 & 1.36 & 6.00 & 1.00\\ \hline
	2012-3 & 40.00 & 2.70 & 2.00 & 2.43 & 13.00 & 1.00\\ \hline
	2012-4 & 3.00 & 1.00 & 1.00 & 0 & 1.00 & 1.00\\ \hline
	Total & 396.00 & 2.26 & 2.00 & 1.82 & 14.00 & 1.00\\ \hline
	 \end{tabular}
  	  \caption{ Descriptive statistic - Throughput - Team two}%
\end{table}
 \newpage 
\subsection{Analyzed data per quarter}
%%skrive at bug er fra da den blir rapportert
In figure \ref{360pq} the average WIP, Throughput, Lead time and Bugs per quarter for team 360 are shown. As stated in section \ref{WIPsec}, to get best possible throughput the WIP-limit should be low. According to the figure \ref{360pq}, when WIP is high, throughput is also high and when WIP is low, throughput also low. This team data is contrary to what is suggested from Kanban and research.






\begin{figure}[!htbp]
\centering
\hspace*{-2in}
\includegraphics[scale=0.8]{Picture/360/360_WIP_TP_BUGS_LT.jpg}
\caption{WIP, Throughput, Lead time and Bugs for 360 per quarter}
\label{360pq} %% 360 per quarter
\end{figure}
\newpage

In figure \ref{Frontendpq} the average WIP, Throughput, Lead time and Bugs per quarter for team Frontend are shown
\begin{figure}[!htbp]
\centering
\hspace*{-1.2in}
\includegraphics[scale=0.7]{Picture/Frontend/Frontend-WIP-TP-LT-BUGS.jpg}
\caption{WIP, Throughput, Lead time and Bugs for Frontend per quarter}
\label{Frontendpq} %% fronted per quarter
\end{figure}
\newpage

In figure \ref{Kryptonpq} the average WIP, Throughput, Lead time and Bugs per quarter for team Krypton are shown
\begin{figure}[!htbp]
\centering
\hspace*{-1.8in}
\includegraphics[scale=0.8]{Picture/Krypton/Krypton-WIP-TP-BUGS-LT.jpg}
\caption{WIP, Throughput, Lead time and Bugs for Krypton per quarter}
\label{Kryptonpq} %% fronted per quarter
\end{figure}
\newpage


In figure \ref{Neonpq} the average WIP, Throughput, Lead time and Bugs per quarter for team Neon are shown
\begin{figure}[!htbp]
\centering
\hspace*{-1.2in}
\includegraphics[scale=1.5]{Picture/Neon/Neon-WIP-TP-BUGS-LT.jpg}
\caption{WIP, Throughput, Lead time and Bugs for Neon per quarter}
\label{Neonpq} %% fronted per quarter
\end{figure}


\chapter*{\centerline{Til Dag}}


\section {360}



  
\begin{table}[htbp]
  \centering
  \begin{adjustwidth}{-2cm}{}
  \scalebox{0.38}{
  \begin{tabular}{ | l | l | l | l | l | l | l | l | l | l | l | l | l | l | l | l | l | }
\hline
Correlations &  &  &  &  &  &  &  &  &  &  &  &  &  &  &  &  \\ \hline
	 &  & Bugs & Throughput & WIP & Churn & TP\_Feature & Churn\_Bugs & Leadtime\_Union & Churn\_Union & Churn\_feature & TP\_bugs & Average\_Days\_Backlog & Bugs\_churn\_average & Average\_ft\_churn & Precent\_bugs\_fininshed & Churn\_Average \\ \hline
	Bugs & Pearson Correlation & 1 & .945** & .753* & .992** & -0.328 & .813** & -0.240 & 0.433 & .730* & .984** & 0.377 & 0.581 & 0.119 & 0.543 & 0.622\\ \hline
	 & Sig. (2-tailed) &  & 0 & 0.012 & 0 & 0.354 & 0.004 & 0.505 & 0.212 & 0.026 & 0 & 0.282 & 0.078 & 0.744 & 0.104 & 0.055\\ \hline
	 & N & 10 & 10 & 10 & 10 & 10 & 10 & 10 & 10 & 9 & 10 & 10 & 10 & 10 & 10 & 10 \\ \hline
	Throughput & Pearson Correlation & .945** & 1 & .816** & .962** & -0.340 & .776** & -0.177 & 0.399 & .736* & .940** & 0.274 & 0.626 & 0.030 & 0.420 & .638* \\ \hline
	 & Sig. (2-tailed) & 0 &  & 0.004 & 0 & 0.336 & 0.008 & 0.624 & 0.253 & 0.024 & 0 & 0.443 & 0.053 & 0.934 & 0.227 & 0.047\\ \hline
	 & N & 10 & 10 & 10 & 10 & 10 & 10 & 10 & 10 & 9 & 10 & 10 & 10 & 10 & 10 & 10 \\ \hline
	WIP & Pearson Correlation & .753* & .816** & 1 & .738* & -0.441 & .708* & -0.336 & 0.365 & 0.467 & .800** & 0.397 & 0.430 & 0.243 & 0.495 & 0.478\\ \hline
	 & Sig. (2-tailed) & 0.012 & 0.004 &  & 0.015 & 0.202 & 0.022 & 0.342 & 0.299 & 0.205 & 0.005 & 0.255 & 0.215 & 0.498 & 0.146 & 0.163\\ \hline
	 & N & 10 & 10 & 10 & 10 & 10 & 10 & 10 & 10 & 9 & 10 & 10 & 10 & 10 & 10 & 10 \\ \hline
	Churn & Pearson Correlation & .992** & .962** & .738* & 1 & -0.317 & .804** & -0.203 & 0.372 & .741* & .974** & 0.349 & 0.581 & 0.087 & 0.475 & 0.617\\ \hline
	 & Sig. (2-tailed) & 0 & 0 & 0.015 &  & 0.372 & 0.005 & 0.574 & 0.289 & 0.022 & 0 & 0.323 & 0.078 & 0.811 & 0.165 & 0.057\\ \hline
	 & N & 10 & 10 & 10 & 10 & 10 & 10 & 10 & 10 & 9 & 10 & 10 & 10 & 10 & 10 & 10 \\ \hline
	TP\_Feature & Pearson Correlation & -0.328 & -0.340 & -0.441 & -0.317 & 1 & -0.419 & 0.376 & -0.460 & -0.208 & -0.360 & 0.097 & -0.171 & -0.295 & -.795** & -0.223\\ \hline
	 & Sig. (2-tailed) & 0.354 & 0.336 & 0.202 & 0.372 &  & 0.228 & 0.284 & 0.181 & 0.591 & 0.307 & 0.789 & 0.637 & 0.408 & 0.006 & 0.536\\ \hline
	 & N & 10 & 10 & 10 & 10 & 10 & 10 & 10 & 10 & 9 & 10 & 10 & 10 & 10 & 10 & 10 \\ \hline
	Churn\_Bugs & Pearson Correlation & .813** & .776** & .708* & .804** & -0.419 & 1 & -0.126 & 0.516 & 0.597 & .884** & .700* & 0.560 & 0.523 & 0.551 & .665* \\ \hline
	 & Sig. (2-tailed) & 0.004 & 0.008 & 0.022 & 0.005 & 0.228 &  & 0.729 & 0.127 & 0.090 & 0.001 & 0.024 & 0.092 & 0.121 & 0.099 & 0.036\\ \hline
	 & N & 10 & 10 & 10 & 10 & 10 & 10 & 10 & 10 & 9 & 10 & 10 & 10 & 10 & 10 & 10 \\ \hline
	Leadtime\_Union & Pearson Correlation & -0.240 & -0.177 & -0.336 & -0.203 & 0.376 & -0.126 & 1 & -0.060 & 0.138 & -0.280 & -0.012 & -0.003 & 0.158 & -0.598 & 0.059\\ \hline
	 & Sig. (2-tailed) & 0.505 & 0.624 & 0.342 & 0.574 & 0.284 & 0.729 &  & 0.868 & 0.724 & 0.433 & 0.974 & 0.994 & 0.663 & 0.068 & 0.871\\ \hline
	 & N & 10 & 10 & 10 & 10 & 10 & 10 & 10 & 10 & 9 & 10 & 10 & 10 & 10 & 10 & 10 \\ \hline
	Churn\_Union & Pearson Correlation & 0.433 & 0.399 & 0.365 & 0.372 & -0.460 & 0.516 & -0.060 & 1 & 0.202 & 0.466 & 0.231 & 0.146 & 0.410 & .634* & 0.198\\ \hline
	 & Sig. (2-tailed) & 0.212 & 0.253 & 0.299 & 0.289 & 0.181 & 0.127 & 0.868 &  & 0.603 & 0.175 & 0.522 & 0.687 & 0.239 & 0.049 & 0.584\\ \hline
	 & N & 10 & 10 & 10 & 10 & 10 & 10 & 10 & 10 & 9 & 10 & 10 & 10 & 10 & 10 & 10 \\ \hline
	Churn\_feature & Pearson Correlation & .730* & .736* & 0.467 & .741* & -0.208 & 0.597 & 0.138 & 0.202 & 1 & .681* & 0.073 & .924** & 0.008 & 0.362 & .947** \\ \hline
	 & Sig. (2-tailed) & 0.026 & 0.024 & 0.205 & 0.022 & 0.591 & 0.090 & 0.724 & 0.603 &  & 0.044 & 0.852 & 0 & 0.984 & 0.338 & 0\\ \hline
	 & N & 9 & 9 & 9 & 9 & 9 & 9 & 9 & 9 & 9 & 9 & 9 & 9 & 9 & 9 & 9 \\ \hline
	TP\_bugs & Pearson Correlation & .984** & .940** & .800** & .974** & -0.360 & .884** & -0.280 & 0.466 & .681* & 1 & 0.494 & 0.574 & 0.192 & 0.568 & 0.620\\ \hline
	 & Sig. (2-tailed) & 0 & 0 & 0.005 & 0 & 0.307 & 0.001 & 0.433 & 0.175 & 0.044 &  & 0.146 & 0.083 & 0.596 & 0.087 & 0.056\\ \hline
	 & N & 10 & 10 & 10 & 10 & 10 & 10 & 10 & 10 & 9 & 10 & 10 & 10 & 10 & 10 & 10 \\ \hline
	Average\_Days\_Backlog & Pearson Correlation & 0.377 & 0.274 & 0.397 & 0.349 & 0.097 & .700* & -0.012 & 0.231 & 0.073 & 0.494 & 1 & 0.101 & 0.595 & 0.130 & 0.217\\ \hline
	 & Sig. (2-tailed) & 0.282 & 0.443 & 0.255 & 0.323 & 0.789 & 0.024 & 0.974 & 0.522 & 0.852 & 0.146 &  & 0.782 & 0.070 & 0.721 & 0.548\\ \hline
	 & N & 10 & 10 & 10 & 10 & 10 & 10 & 10 & 10 & 9 & 10 & 10 & 10 & 10 & 10 & 10 \\ \hline
	Bugs\_churn\_average & Pearson Correlation & 0.581 & 0.626 & 0.430 & 0.581 & -0.171 & 0.560 & -0.003 & 0.146 & .924** & 0.574 & 0.101 & 1 & -0.127 & 0.322 & .973** \\ \hline
	 & Sig. (2-tailed) & 0.078 & 0.053 & 0.215 & 0.078 & 0.637 & 0.092 & 0.994 & 0.687 & 0 & 0.083 & 0.782 &  & 0.726 & 0.365 & 0\\ \hline
	 & N & 10 & 10 & 10 & 10 & 10 & 10 & 10 & 10 & 9 & 10 & 10 & 10 & 10 & 10 & 10 \\ \hline
	Average\_ft\_churn & Pearson Correlation & 0.119 & 0.030 & 0.243 & 0.087 & -0.295 & 0.523 & 0.158 & 0.410 & 0.008 & 0.192 & 0.595 & -0.127 & 1 & 0.291 & 0.088\\ \hline
	 & Sig. (2-tailed) & 0.744 & 0.934 & 0.498 & 0.811 & 0.408 & 0.121 & 0.663 & 0.239 & 0.984 & 0.596 & 0.070 & 0.726 &  & 0.415 & 0.808\\ \hline
	 & N & 10 & 10 & 10 & 10 & 10 & 10 & 10 & 10 & 9 & 10 & 10 & 10 & 10 & 10 & 10 \\ \hline
	Precent\_bugs\_fininshed & Pearson Correlation & 0.543 & 0.420 & 0.495 & 0.475 & -.795** & 0.551 & -0.598 & .634* & 0.362 & 0.568 & 0.130 & 0.322 & 0.291 & 1 & 0.369\\ \hline
	 & Sig. (2-tailed) & 0.104 & 0.227 & 0.146 & 0.165 & 0.006 & 0.099 & 0.068 & 0.049 & 0.338 & 0.087 & 0.721 & 0.365 & 0.415 &  & 0.294\\ \hline
	 & N & 10 & 10 & 10 & 10 & 10 & 10 & 10 & 10 & 9 & 10 & 10 & 10 & 10 & 10 & 10 \\ \hline
	Churn\_Average & Pearson Correlation & 0.622 & .638* & 0.478 & 0.617 & -0.223 & .665* & 0.059 & 0.198 & .947** & 0.620 & 0.217 & .973** & 0.088 & 0.369 & 1 \\ \hline
	 & Sig. (2-tailed) & 0.055 & 0.047 & 0.163 & 0.057 & 0.536 & 0.036 & 0.871 & 0.584 & 0 & 0.056 & 0.548 & 0 & 0.808 & 0.294 &  \\ \hline
	 & N & 10 & 10 & 10 & 10 & 10 & 10 & 10 & 10 & 9 & 10 & 10 & 10 & 10 & 10 & 10 \\ \hline
	 \end{tabular}
 }
  \centerline{  ** Correlation is significant at the 0.01 level (2-tailed).}
    \centerline {* Correlation is significant at the 0.05 level (2-tailed).}
      \caption{360 - Correlation}
  \label{tab:addlabel}%
 \end{adjustwidth}
\end{table}%
\begin{table}[!htbp]
\centering
  \begin{adjustwidth}{-1cm}{}
     \scalebox{0.75}{
\begin{tabular}{ | l | l | l | l | l | }
\hline
	 &  &  &  &  \\ \hline
	Team & Quarter &  & Churn & Leadtime \\ \hline
	360 & 2010-3 & N & 11 & 11 \\ \hline
	 &  & Mean & 17 & 3.55 \\ \hline
	 &  & Median & 11 & 3 \\ \hline
	 &  & Std. Deviation & 18.93 & 2.50 \\ \hline
	 &  & Maximum & 60 & 8 \\ \hline
	 &  & Minimum & 0 & 1 \\ \hline
	 &  &  &  &  \\ \hline
	 & 2010-4 & N & 3 & 3 \\ \hline
	 &  & Mean & 0.67 & 1.67 \\ \hline
	 &  & Median & 1 & 2 \\ \hline
	 &  & Std. Deviation & 0.57 & 0.57 \\ \hline
	 &  & Maximum & 1 & 2 \\ \hline
	 &  & Minimum & 0 & 1 \\ \hline
	 &  & N & 136 & 136 \\ \hline
	 & 2011-2 & Mean & 14.4 & 2.7 \\ \hline
	 &  & Median & 3 & 2 \\ \hline
	 &  & Std. Deviation & 21.51 & 1.85 \\ \hline
	 &  & Maximum & 83 & 8 \\ \hline
	 &  & Minimum & 0 & 1 \\ \hline
	 &  & N & 48 & 48 \\ \hline
	 & 2011-3 & Mean & 1.67 & 3.06 \\ \hline
	 &  & Median & 0 & 2.5 \\ \hline
	 &  & Std. Deviation & 7.71 & 2.13 \\ \hline
	 &  & Maximum & 50 & 8 \\ \hline
	 &  & Minimum & 0 & 1 \\ \hline
	 &  & N & 156 & 156 \\ \hline
	 & 2011-4 & Mean & 14.81 & 2.25 \\ \hline
	 &  & Median & 6.5 & 2 \\ \hline
	 &  & Std. Deviation & 19.90 & 1.53 \\ \hline
	 &  & Maximum & 86 & 8 \\ \hline
	 &  & Minimum & 0 & 1 \\ \hline
	 &  & N & 157 & 157 \\ \hline
	 & 2012-1 & Mean & 9.47 & 2.68 \\ \hline
	 &  & Median & 0 & 2 \\ \hline
	 &  & Std. Deviation & 19.54 & 1.758 \\ \hline
	 &  & Maximum & 86 & 8 \\ \hline
	 &  & Minimum & 0 & 1 \\ \hline
	 &  & N & 238 & 238 \\ \hline
	 & 2012-2 & Mean & 10.11 & 2.78 \\ \hline
	 &  & Median & 0 & 2 \\ \hline
	 &  & Std. Deviation & 18.50 & 1.597 \\ \hline
	 &  & Maximum & 83 & 8 \\ \hline
	 &  & Minimum & 0 & 1 \\ \hline
	 &  & N & 53 & 53 \\ \hline
	 & 2012-3 & Mean & 3.02 & 1.68 \\ \hline
	 &  & Median & 0 & 1 \\ \hline
	 &  & Std. Deviation & 9.61 & 1.18\\ \hline
	 &  & Maximum & 53 & 7 \\ \hline
	 &  & Minimum & 0 & 1 \\ \hline
	 &  & N & 26 & 26 \\ \hline
	 & 2012-4 & Mean & 21.42 & 2.08 \\ \hline
	 &  & Median & 14.5 & 2 \\ \hline
	 &  & Std. Deviation & 20.54 & 1.16 \\ \hline
	 &  & Maximum & 67 & 5 \\ \hline
	 &  & Minimum & 1 & 1 \\ \hline
	 &  & N & 902 & 902 \\ \hline
	 & Total & Mean & 10.14 & 2.7 \\ \hline
	 &  & Median & 0 & 2 \\ \hline
	 &  & Std. Deviation & 18.55 & 1.798 \\ \hline
	 &  & Maximum & 86 & 8 \\ \hline
	 &  & Minimum & 0 & 1 \\ \hline
\end{tabular}
}
\caption{Descriptive statistic  - Lead-time and churn - 360}
\end{adjustwidth}
 \end{table}%
 
	

 
%%%%%%%%%%%%%%%%%
\section*{Frontend} 
 \begin{table}[!htbp]
  \centering
    \begin{adjustwidth}{-2.5cm}{}
    \scalebox{0.38}{
    \begin{tabular}{ | l | l | l | l | l | l | l | l | l | l | l | l | l | l | l | l | l | }
\hline
	Correlations &  &  &  &  &  &  &  &  &  &  &  &  &  &  &  &  \\ \hline
	 &  & WIP & Throughput & Bugs & Churn & TP\_feature & Churn\_Bugs & Churn\_union & Leadtime\_union & Churn\_feature & Tp\_bugs & Average\_Days\_Backlog\_Bugs & Bugs\_Churn\_average & Churn\_feature\_Average & Precent\_bugs\_fininshed & Churn\_Average \\ \hline
	WIP & Pearson Correlation & 1 & 0.491 & 0.278 & -0.446 & 0.380 & -0.446 & 0.316 & 0.096 & 0.362 & 0.508 & -0.178 & 0.257 & 0.337 & 0.367 & 0.223\\ \hline
	 & Sig. (2-tailed) &  & 0.150 & 0.436 & 0.196 & 0.279 & 0.196 & 0.373 & 0.791 & 0.304 & 0.134 & 0.622 & 0.473 & 0.341 & 0.297 & 0.537\\ \hline
	 & N & 10 & 10 & 10 & 10 & 10 & 10 & 10 & 10 & 10 & 10 & 10 & 10 & 10 & 10 & 10 \\ \hline
	Throughput & Pearson Correlation & 0.491 & 1 & .773** & -0.036 & .741* & 0.217 & -0.494 & -0.539 & -0.433 & .758* & 0.055 & 0.259 & -0.609 & .745* & -0.575\\ \hline
	 & Sig. (2-tailed) & 0.150 &  & 0.009 & 0.921 & 0.014 & 0.548 & 0.147 & 0.108 & 0.212 & 0.011 & 0.879 & 0.470 & 0.062 & 0.013 & 0.082\\ \hline
	 & N & 10 & 10 & 10 & 10 & 10 & 10 & 10 & 10 & 10 & 10 & 10 & 10 & 10 & 10 & 10 \\ \hline
	Bugs & Pearson Correlation & 0.278 & .773** & 1 & 0.259 & 0.534 & 0.398 & -0.555 & -0.581 & -0.478 & 0.271 & 0.005 & 0.296 & -0.571 & 0.498 & -0.624\\ \hline
	 & Sig. (2-tailed) & 0.436 & 0.009 &  & 0.471 & 0.111 & 0.254 & 0.096 & 0.078 & 0.162 & 0.449 & 0.988 & 0.407 & 0.085 & 0.143 & 0.054\\ \hline
	 & N & 10 & 10 & 10 & 10 & 10 & 10 & 10 & 10 & 10 & 10 & 10 & 10 & 10 & 10 & 10 \\ \hline
	Churn & Pearson Correlation & -0.446 & -0.036 & 0.259 & 1 & -0.219 & .791** & -0.522 & -0.449 & -0.517 & -0.110 & -0.174 & 0.172 & -0.485 & -0.014 & -0.471\\ \hline
	 & Sig. (2-tailed) & 0.196 & 0.921 & 0.471 &  & 0.543 & 0.006 & 0.122 & 0.192 & 0.126 & 0.762 & 0.631 & 0.634 & 0.155 & 0.970 & 0.170\\ \hline
	 & N & 10 & 10 & 10 & 10 & 10 & 10 & 10 & 10 & 10 & 10 & 10 & 10 & 10 & 10 & 10 \\ \hline
	TP\_feature & Pearson Correlation & 0.380 & .741* & 0.534 & -0.219 & 1 & 0.214 & -0.244 & -0.312 & -0.026 & 0.427 & -0.256 & -0.007 & -0.294 & .640* & -0.319\\ \hline
	 & Sig. (2-tailed) & 0.279 & 0.014 & 0.111 & 0.543 &  & 0.553 & 0.497 & 0.381 & 0.943 & 0.219 & 0.475 & 0.984 & 0.409 & 0.046 & 0.369\\ \hline
	 & N & 10 & 10 & 10 & 10 & 10 & 10 & 10 & 10 & 10 & 10 & 10 & 10 & 10 & 10 & 10 \\ \hline
	Churn\_Bugs & Pearson Correlation & -0.446 & 0.217 & 0.398 & .791** & 0.214 & 1 & -.707* & -.771** & -0.460 & 0.040 & -0.378 & -0.035 & -0.590 & 0.388 & -.689* \\ \hline
	 & Sig. (2-tailed) & 0.196 & 0.548 & 0.254 & 0.006 & 0.553 &  & 0.022 & 0.009 & 0.181 & 0.912 & 0.282 & 0.924 & 0.073 & 0.267 & 0.027\\ \hline
	 & N & 10 & 10 & 10 & 10 & 10 & 10 & 10 & 10 & 10 & 10 & 10 & 10 & 10 & 10 & 10 \\ \hline
	Churn\_union & Pearson Correlation & 0.316 & -0.494 & -0.555 & -0.522 & -0.244 & -.707* & 1 & .704* & .844** & -0.368 & -0.139 & 0.152 & .841** & -0.312 & .981** \\ \hline
	 & Sig. (2-tailed) & 0.373 & 0.147 & 0.096 & 0.122 & 0.497 & 0.022 &  & 0.023 & 0.002 & 0.296 & 0.702 & 0.674 & 0.002 & 0.379 & 0\\ \hline
	 & N & 10 & 10 & 10 & 10 & 10 & 10 & 10 & 10 & 10 & 10 & 10 & 10 & 10 & 10 & 10 \\ \hline
	Leadtime\_union & Pearson Correlation & 0.096 & -0.539 & -0.581 & -0.449 & -0.312 & -.771** & .704* & 1 & 0.492 & -0.301 & 0.327 & -0.227 & 0.631 & -.697* & .766** \\ \hline
	 & Sig. (2-tailed) & 0.791 & 0.108 & 0.078 & 0.192 & 0.381 & 0.009 & 0.023 &  & 0.148 & 0.398 & 0.357 & 0.528 & 0.050 & 0.025 & 0.010\\ \hline
	 & N & 10 & 10 & 10 & 10 & 10 & 10 & 10 & 10 & 10 & 10 & 10 & 10 & 10 & 10 & 10 \\ \hline
	Churn\_feature & Pearson Correlation & 0.362 & -0.433 & -0.478 & -0.517 & -0.026 & -0.460 & .844** & 0.492 & 1 & -0.314 & -0.358 & -0.201 & .921** & -0.172 & .842** \\ \hline
	 & Sig. (2-tailed) & 0.304 & 0.212 & 0.162 & 0.126 & 0.943 & 0.181 & 0.002 & 0.148 &  & 0.377 & 0.310 & 0.579 & 0 & 0.635 & 0.002\\ \hline
	 & N & 10 & 10 & 10 & 10 & 10 & 10 & 10 & 10 & 10 & 10 & 10 & 10 & 10 & 10 & 10 \\ \hline
	Tp\_bugs & Pearson Correlation & 0.508 & .758* & 0.271 & -0.110 & 0.427 & 0.040 & -0.368 & -0.301 & -0.314 & 1 & 0.096 & 0.034 & -0.393 & 0.622 & -0.406\\ \hline
	 & Sig. (2-tailed) & 0.134 & 0.011 & 0.449 & 0.762 & 0.219 & 0.912 & 0.296 & 0.398 & 0.377 &  & 0.792 & 0.926 & 0.261 & 0.055 & 0.245\\ \hline
	 & N & 10 & 10 & 10 & 10 & 10 & 10 & 10 & 10 & 10 & 10 & 10 & 10 & 10 & 10 & 10 \\ \hline
	Average\_Days\_Backlog\_Bugs & Pearson Correlation & -0.178 & 0.055 & 0.005 & -0.174 & -0.256 & -0.378 & -0.139 & 0.327 & -0.358 & 0.096 & 1 & -0.267 & -0.272 & -0.489 & -0.035\\ \hline
	 & Sig. (2-tailed) & 0.622 & 0.879 & 0.988 & 0.631 & 0.475 & 0.282 & 0.702 & 0.357 & 0.310 & 0.792 &  & 0.456 & 0.446 & 0.151 & 0.925\\ \hline
	 & N & 10 & 10 & 10 & 10 & 10 & 10 & 10 & 10 & 10 & 10 & 10 & 10 & 10 & 10 & 10 \\ \hline
	Bugs\_Churn\_average & Pearson Correlation & 0.257 & 0.259 & 0.296 & 0.172 & -0.007 & -0.035 & 0.152 & -0.227 & -0.201 & 0.034 & -0.267 & 1 & -0.197 & 0.371 & 0.016\\ \hline
	 & Sig. (2-tailed) & 0.473 & 0.470 & 0.407 & 0.634 & 0.984 & 0.924 & 0.674 & 0.528 & 0.579 & 0.926 & 0.456 &  & 0.585 & 0.291 & 0.966\\ \hline
	 & N & 10 & 10 & 10 & 10 & 10 & 10 & 10 & 10 & 10 & 10 & 10 & 10 & 10 & 10 & 10 \\ \hline
	Churn\_feature\_Average & Pearson Correlation & 0.337 & -0.609 & -0.571 & -0.485 & -0.294 & -0.590 & .841** & 0.631 & .921** & -0.393 & -0.272 & -0.197 & 1 & -0.389 & .846** \\ \hline
	 & Sig. (2-tailed) & 0.341 & 0.062 & 0.085 & 0.155 & 0.409 & 0.073 & 0.002 & 0.050 & 0 & 0.261 & 0.446 & 0.585 &  & 0.267 & 0.002\\ \hline
	 & N & 10 & 10 & 10 & 10 & 10 & 10 & 10 & 10 & 10 & 10 & 10 & 10 & 10 & 10 & 10 \\ \hline
	Precent\_bugs\_fininshed & Pearson Correlation & 0.367 & .745* & 0.498 & -0.014 & .640* & 0.388 & -0.312 & -.697* & -0.172 & 0.622 & -0.489 & 0.371 & -0.389 & 1 & -0.445\\ \hline
	 & Sig. (2-tailed) & 0.297 & 0.013 & 0.143 & 0.970 & 0.046 & 0.267 & 0.379 & 0.025 & 0.635 & 0.055 & 0.151 & 0.291 & 0.267 &  & 0.197\\ \hline
	 & N & 10 & 10 & 10 & 10 & 10 & 10 & 10 & 10 & 10 & 10 & 10 & 10 & 10 & 10 & 10 \\ \hline
	Churn\_Average & Pearson Correlation & 0.223 & -0.575 & -0.624 & -0.471 & -0.319 & -.689* & .981** & .766** & .842** & -0.406 & -0.035 & 0.016 & .846** & -0.445 & 1 \\ \hline
	 & Sig. (2-tailed) & 0.537 & 0.082 & 0.054 & 0.170 & 0.369 & 0.027 & 0 & 0.010 & 0.002 & 0.245 & 0.925 & 0.966 & 0.002 & 0.197 &  \\ \hline
	 & N & 10 & 10 & 10 & 10 & 10 & 10 & 10 & 10 & 10 & 10 & 10 & 10 & 10 & 10 & 10 \\ \hline
\end{tabular}
    }
       \centerline{  ** Correlation is significant at the 0.01 level (2-tailed).}
    \centerline {* Correlation is significant at the 0.05 level (2-tailed).}
      \caption{Frontend Correlation}
  \label{tab:addlabel}%
   \end{adjustwidth}
\end{table}%

\begin{table}[!htbp]
\centering
  \begin{adjustwidth}{-1cm}{}
     \scalebox{0.75}{
\begin{tabular}{ | l | l | l | l | l | } 
\hline
	Team & Quarter &  & Leadtime & Churn \\ \hline
	Frontend & 2010-3 & N & 58 & 58 \\ \hline
	 &  & Mean & 7.43 & 127.93 \\ \hline
	 &  & Median & 6 & 53.5 \\ \hline
	 &  & Std. Deviation & 5.567 & 187.011 \\ \hline
	 &  & Maximum & 22 & 1072 \\ \hline
	 &  & Minimum & 1 & 0 \\ \hline
	 & 2010-4 & N & 130 & 130 \\ \hline
	 &  & Mean & 10.32 & 158.72 \\ \hline
	 &  & Median & 4 & 14 \\ \hline
	 &  & Std. Deviation & 15.30 & 444.805 \\ \hline
	 &  & Maximum & 78 & 4308 \\ \hline
	 &  & Minimum & 1 & 0 \\ \hline
	 & 2011-1 & N & 95 & 95 \\ \hline
	 &  & Mean & 28.01 & 566.73 \\ \hline
	 &  & Median & 11 & 47 \\ \hline
	 &  & Std. Deviation & 52.183 & 3696.125 \\ \hline
	 &  & Maximum & 434 & 35917 \\ \hline
	 &  & Minimum & 1 & 0 \\ \hline
	 & 2011-2 & N & 132 & 132 \\ \hline
	 &  & Mean & 12.17 & 496.8 \\ \hline
	 &  & Median & 5 & 19 \\ \hline
	 &  & Std. Deviation & 23.548 & 3240.172 \\ \hline
	 &  & Maximum & 184 & 35956 \\ \hline
	 &  & Minimum & 1 & 0 \\ \hline
	 & 2011-3 & N & 146 & 146 \\ \hline
	 &  & Mean & 10.66 & 174.07 \\ \hline
	 &  & Median & 6 & 14 \\ \hline
	 &  & Std. Deviation & 12.577 & 655.578 \\ \hline
	 &  & Maximum & 79 & 5650 \\ \hline
	 &  & Minimum & 1 & 0 \\ \hline
	 & 2011-4 & N & 167 & 167 \\ \hline
	 &  & Mean & 9.85 & 110.98 \\ \hline
	 &  & Median & 7 & 3 \\ \hline
	 &  & Std. Deviation & 10.9 & 480.32 \\ \hline
	 &  & Maximum & 62 & 5672 \\ \hline
	 &  & Minimum & 1 & 0 \\ \hline
	 & 2012-1 & N & 188 & 188 \\ \hline
	 &  & Mean & 13.9 & 118.64 \\ \hline
	 &  & Median & 8 & 5.5 \\ \hline
	 &  & Std. Deviation & 21.92 & 568.03 \\ \hline
	 &  & Maximum & 150 & 7565 \\ \hline
	 &  & Minimum & 1 & 0 \\ \hline
	 & 2012-2 & N & 77 & 77 \\ \hline
	 &  & Mean & 22.53 & 849.42 \\ \hline
	 &  & Median & 17 & 15 \\ \hline
	 &  & Std. Deviation & 25.52 & 3575.50 \\ \hline
	 &  & Maximum & 106 & 27170 \\ \hline
	 &  & Minimum & 1 & 0 \\ \hline
	 & 2012-3 & N & 93 & 93 \\ \hline
	 &  & Mean & 13.76 & 458.92 \\ \hline
	 &  & Median & 8 & 25 \\ \hline
	 &  & Std. Deviation & 18.023 & 1272.32 \\ \hline
	 &  & Maximum & 107 & 8600 \\ \hline
	 &  & Minimum & 1 & 0 \\ \hline
	 & 2012-4 & N & 82 & 82 \\ \hline
	 &  & Mean & 8.8007 & 428.67 \\ \hline
	 &  & Median & 4 & 54 \\ \hline
	 &  & Std. Deviation & 14.978& 1178.471 \\ \hline
	 &  & Maximum & 83 & 9368 \\ \hline
	 &  & Minimum & 1 & 0 \\ \hline
	 & Total & N & 1168 & 1168 \\ \hline
	 &  & Mean & 13.35 & 305.62 \\ \hline
	 &  & Median & 7 & 15 \\ \hline
	 &  & Std. Deviation & 23.15 & 1883.16 \\ \hline
	 &  & Maximum & 434 & 35956 \\ \hline
	 &  & Minimum & 1 & 0 \\ \hline
\end{tabular}
}
\caption{Descriptive statistic  - Lead-time and churn - Frontend}
\end{adjustwidth}
 \end{table}%
 

 

 \begin{table}[!htbp]
\begin{tabular}{ | l | l | l | l | l | l | l | l | }
\hline
TeamName & Quarter & N & Mean & Median & Std. Deviation & Maximum & Minimum \\ \hline
Frontend & 2010-3 & 14 & 2.79 & 2 & 2.91 & 12 & 1 \\ \hline
	 & 2010-4 & 39 & 2 & 1 & 1.53 & 8 & 1 \\ \hline
	 & 2011-1 & 20 & 1.85 & 1 & 1.226 & 5 & 1 \\ \hline
	 & 2011-2 & 29 & 2.86 & 2 & 3.21 & 16 & 1 \\ \hline
	 & 2011-3 & 37 & 2.41 & 2 & 1.93 & 10 & 1 \\ \hline
	 & 2011-4 & 35 & 2.89 & 1 & 5.02 & 30 & 1 \\ \hline
	 & 2012-1 & 39 & 3.56 & 2 & 3.90& 23 & 1 \\ \hline
	 & 2012-2 & 31 & 1.81 & 1 & 1.4 & 7 & 1 \\ \hline
	 & 2012-3 & 24 & 2.54 & 2 & 1.79 & 7 & 1 \\ \hline
	 & 2012-4 & 14 & 2.86 & 2 & 1.79 & 7 & 1 \\ \hline
	 & Total & 282 & 2.56 & 2 & 2.895 & 30 & 1 \\ \hline
	 	 \end{tabular}
  	  \caption{Frontend - Descriptive statistic - Bugs }%
\end{table}

 \begin{table}[!htbp]
\begin{tabular}{ | l | l | l | l | l | l | l | l | }
\hline
TeamName & Quarter & N & Mean & Median & Std. Deviation & Maximum & Minimum \\ \hline
Frontend & 2010-3 & 8 & 1.25 & 1 & 0.46 & 2 & 1 \\ \hline
	 & 2010-4 & 35 & 1.63 & 1 & 0.877 & 4 & 1 \\ \hline
	 & 2011-1 & 32 & 1.53 & 1 & 0.67& 3 & 1 \\ \hline
	 & 2011-2 & 31 & 1.74 & 2 & 0.89 & 4 & 1 \\ \hline
	 & 2011-3 & 31 & 1.97 & 2 & 1.25& 6 & 1 \\ \hline
	 & 2011-4 & 28 & 1.93 & 1 & 1.94& 11 & 1 \\ \hline
	 & 2012-1 & 22 & 2.18& 1 & 2.01 & 7 & 1 \\ \hline
	 & 2012-2 & 23 & 1.48 & 1 & 1.123 & 6 & 1 \\ \hline
	 & 2012-3 & 21 & 1.48 & 1 & 0.873 & 4 & 1 \\ \hline
	 & 2012-4 & 19 & 2.47 & 2 & 1.679 & 8 & 1 \\ \hline
	 & Total & 250 & 1.78 & 1 & 1.294 & 11 & 1 \\ \hline
 	 \end{tabular}
  	  \caption{Frontend - Descriptive statistic - TP feature }%
\end{table}

 \begin{table}[!htbp]
\begin{tabular}{ | l | l | l | l | l | l | l | l | }
\hline
TeamName & Quarter & N & Mean & Median & Std. Deviation & Maximum & Minimum \\ \hline
	Frontend & 2010-3 & 15 & 2 & 2 & 1.363 & 6 & 1 \\ \hline
	 & 2010-4 & 41 & 1.98 & 2 & 1.387 & 8 & 1 \\ \hline
	 & 2011-1 & 18 & 1.33 & 1 & 0.48 & 2 & 1 \\ \hline
	 & 2011-2 & 35 & 2.50 & 2 & 2.161 & 10 & 1 \\ \hline
	 & 2011-3 & 38 & 2.18 & 2 & 1.60 & 8 & 1 \\ \hline
	 & 2011-4 & 40 & 2.6 & 2 & 2.52 & 14 & 1 \\ \hline
	 & 2012-1 & 41 & 3.44 & 3 & 2.61 & 10 & 1 \\ \hline
	 & 2012-2 & 29 & 1.9 & 1 & 1.472 & 7 & 1 \\ \hline
	 & 2012-3 & 30 & 2.13 & 1.5 & 1.69 & 8 & 1 \\ \hline
	 & 2012-4 & 25 & 2.12 & 1 & 2.10 & 10 & 1 \\ \hline
	 & Total & 312 & 2.31 & 2 & 1.982 & 14 & 1 \\ \hline
 	 \end{tabular}
  	  \caption{Frontend - Descriptive statistic - TP bugs }%
\end{table}
%%%%%%%%%%%%%%

\section{Krypton}
\begin{table}[!htbp]
  \centering
   \begin{adjustwidth}{-2.5cm}{}
    \scalebox{0.38}{
    \begin{tabular}{|l|l|l|l|l|l|l|l|l|}
    \hline
      \begin{tabular}{ | l | l | l | l | l | l | l | l | l | l | l | l | l | l | l | l | l | }
\hline
	Correlations &  &  &  &  &  &  &  &  &  &  &  &  &  &  &  &  \\ \hline
	 &  & WIP & Throughput & Bugs & Churn & TP\_feature & Churn\_Bugs & Churn\_union & Leadtime\_union & Churn\_feature & Tp\_bugs & Average\_Days\_Backlog\_Bugs & Bugs\_Churn\_average & Churn\_feature\_Average & Precent\_bugs\_fininshed & Churn\_Average \\ \hline
	WIP & Pearson Correlation & 1 & 0.491 & 0.278 & -0.446 & 0.380 & -0.446 & 0.316 & 0.096 & 0.362 & 0.508 & -0.178 & 0.257 & 0.337 & 0.367 & 0.223\\ \hline
	 & Sig. (2-tailed) &  & 0.150 & 0.436 & 0.196 & 0.279 & 0.196 & 0.373 & 0.791 & 0.304 & 0.134 & 0.622 & 0.473 & 0.341 & 0.297 & 0.537\\ \hline
	 & N & 10 & 10 & 10 & 10 & 10 & 10 & 10 & 10 & 10 & 10 & 10 & 10 & 10 & 10 & 10 \\ \hline
	Throughput & Pearson Correlation & 0.491 & 1 & .773** & -0.036 & .741* & 0.217 & -0.494 & -0.539 & -0.433 & .758* & 0.055 & 0.259 & -0.609 & .745* & -0.575\\ \hline
	 & Sig. (2-tailed) & 0.150 &  & 0.009 & 0.921 & 0.014 & 0.548 & 0.147 & 0.108 & 0.212 & 0.011 & 0.879 & 0.470 & 0.062 & 0.013 & 0.082\\ \hline
	 & N & 10 & 10 & 10 & 10 & 10 & 10 & 10 & 10 & 10 & 10 & 10 & 10 & 10 & 10 & 10 \\ \hline
	Bugs & Pearson Correlation & 0.278 & .773** & 1 & 0.259 & 0.534 & 0.398 & -0.555 & -0.581 & -0.478 & 0.271 & 0.005 & 0.296 & -0.571 & 0.498 & -0.624\\ \hline
	 & Sig. (2-tailed) & 0.436 & 0.009 &  & 0.471 & 0.111 & 0.254 & 0.096 & 0.078 & 0.162 & 0.449 & 0.988 & 0.407 & 0.085 & 0.143 & 0.054\\ \hline
	 & N & 10 & 10 & 10 & 10 & 10 & 10 & 10 & 10 & 10 & 10 & 10 & 10 & 10 & 10 & 10 \\ \hline
	Churn & Pearson Correlation & -0.446 & -0.036 & 0.259 & 1 & -0.219 & .791** & -0.522 & -0.449 & -0.517 & -0.110 & -0.174 & 0.172 & -0.485 & -0.014 & -0.471\\ \hline
	 & Sig. (2-tailed) & 0.196 & 0.921 & 0.471 &  & 0.543 & 0.006 & 0.122 & 0.192 & 0.126 & 0.762 & 0.631 & 0.634 & 0.155 & 0.970 & 0.170\\ \hline
	 & N & 10 & 10 & 10 & 10 & 10 & 10 & 10 & 10 & 10 & 10 & 10 & 10 & 10 & 10 & 10 \\ \hline
	TP\_feature & Pearson Correlation & 0.380 & .741* & 0.534 & -0.219 & 1 & 0.214 & -0.244 & -0.312 & -0.026 & 0.427 & -0.256 & -0.007 & -0.294 & .640* & -0.319\\ \hline
	 & Sig. (2-tailed) & 0.279 & 0.014 & 0.111 & 0.543 &  & 0.553 & 0.497 & 0.381 & 0.943 & 0.219 & 0.475 & 0.984 & 0.409 & 0.046 & 0.369\\ \hline
	 & N & 10 & 10 & 10 & 10 & 10 & 10 & 10 & 10 & 10 & 10 & 10 & 10 & 10 & 10 & 10 \\ \hline
	Churn\_Bugs & Pearson Correlation & -0.446 & 0.217 & 0.398 & .791** & 0.214 & 1 & -.707* & -.771** & -0.460 & 0.040 & -0.378 & -0.035 & -0.590 & 0.388 & -.689* \\ \hline
	 & Sig. (2-tailed) & 0.196 & 0.548 & 0.254 & 0.006 & 0.553 &  & 0.022 & 0.009 & 0.181 & 0.912 & 0.282 & 0.924 & 0.073 & 0.267 & 0.027\\ \hline
	 & N & 10 & 10 & 10 & 10 & 10 & 10 & 10 & 10 & 10 & 10 & 10 & 10 & 10 & 10 & 10 \\ \hline
	Churn\_union & Pearson Correlation & 0.316 & -0.494 & -0.555 & -0.522 & -0.244 & -.707* & 1 & .704* & .844** & -0.368 & -0.139 & 0.152 & .841** & -0.312 & .981** \\ \hline
	 & Sig. (2-tailed) & 0.373 & 0.147 & 0.096 & 0.122 & 0.497 & 0.022 &  & 0.023 & 0.002 & 0.296 & 0.702 & 0.674 & 0.002 & 0.379 & 0\\ \hline
	 & N & 10 & 10 & 10 & 10 & 10 & 10 & 10 & 10 & 10 & 10 & 10 & 10 & 10 & 10 & 10 \\ \hline
	Leadtime\_union & Pearson Correlation & 0.096 & -0.539 & -0.581 & -0.449 & -0.312 & -.771** & .704* & 1 & 0.492 & -0.301 & 0.327 & -0.227 & 0.631 & -.697* & .766** \\ \hline
	 & Sig. (2-tailed) & 0.791 & 0.108 & 0.078 & 0.192 & 0.381 & 0.009 & 0.023 &  & 0.148 & 0.398 & 0.357 & 0.528 & 0.050 & 0.025 & 0.010\\ \hline
	 & N & 10 & 10 & 10 & 10 & 10 & 10 & 10 & 10 & 10 & 10 & 10 & 10 & 10 & 10 & 10 \\ \hline
	Churn\_feature & Pearson Correlation & 0.362 & -0.433 & -0.478 & -0.517 & -0.026 & -0.460 & .844** & 0.492 & 1 & -0.314 & -0.358 & -0.201 & .921** & -0.172 & .842** \\ \hline
	 & Sig. (2-tailed) & 0.304 & 0.212 & 0.162 & 0.126 & 0.943 & 0.181 & 0.002 & 0.148 &  & 0.377 & 0.310 & 0.579 & 0 & 0.635 & 0.002\\ \hline
	 & N & 10 & 10 & 10 & 10 & 10 & 10 & 10 & 10 & 10 & 10 & 10 & 10 & 10 & 10 & 10 \\ \hline
	Tp\_bugs & Pearson Correlation & 0.508 & .758* & 0.271 & -0.110 & 0.427 & 0.040 & -0.368 & -0.301 & -0.314 & 1 & 0.096 & 0.034 & -0.393 & 0.622 & -0.406\\ \hline
	 & Sig. (2-tailed) & 0.134 & 0.011 & 0.449 & 0.762 & 0.219 & 0.912 & 0.296 & 0.398 & 0.377 &  & 0.792 & 0.926 & 0.261 & 0.055 & 0.245\\ \hline
	 & N & 10 & 10 & 10 & 10 & 10 & 10 & 10 & 10 & 10 & 10 & 10 & 10 & 10 & 10 & 10 \\ \hline
	Average\_Days\_Backlog\_Bugs & Pearson Correlation & -0.178 & 0.055 & 0.005 & -0.174 & -0.256 & -0.378 & -0.139 & 0.327 & -0.358 & 0.096 & 1 & -0.267 & -0.272 & -0.489 & -0.035\\ \hline
	 & Sig. (2-tailed) & 0.622 & 0.879 & 0.988 & 0.631 & 0.475 & 0.282 & 0.702 & 0.357 & 0.310 & 0.792 &  & 0.456 & 0.446 & 0.151 & 0.925\\ \hline
	 & N & 10 & 10 & 10 & 10 & 10 & 10 & 10 & 10 & 10 & 10 & 10 & 10 & 10 & 10 & 10 \\ \hline
	Bugs\_Churn\_average & Pearson Correlation & 0.257 & 0.259 & 0.296 & 0.172 & -0.007 & -0.035 & 0.152 & -0.227 & -0.201 & 0.034 & -0.267 & 1 & -0.197 & 0.371 & 0.016\\ \hline
	 & Sig. (2-tailed) & 0.473 & 0.470 & 0.407 & 0.634 & 0.984 & 0.924 & 0.674 & 0.528 & 0.579 & 0.926 & 0.456 &  & 0.585 & 0.291 & 0.966\\ \hline
	 & N & 10 & 10 & 10 & 10 & 10 & 10 & 10 & 10 & 10 & 10 & 10 & 10 & 10 & 10 & 10 \\ \hline
	Churn\_feature\_Average & Pearson Correlation & 0.337 & -0.609 & -0.571 & -0.485 & -0.294 & -0.590 & .841** & 0.631 & .921** & -0.393 & -0.272 & -0.197 & 1 & -0.389 & .846** \\ \hline
	 & Sig. (2-tailed) & 0.341 & 0.062 & 0.085 & 0.155 & 0.409 & 0.073 & 0.002 & 0.050 & 0 & 0.261 & 0.446 & 0.585 &  & 0.267 & 0.002\\ \hline
	 & N & 10 & 10 & 10 & 10 & 10 & 10 & 10 & 10 & 10 & 10 & 10 & 10 & 10 & 10 & 10 \\ \hline
	Precent\_bugs\_fininshed & Pearson Correlation & 0.367 & .745* & 0.498 & -0.014 & .640* & 0.388 & -0.312 & -.697* & -0.172 & 0.622 & -0.489 & 0.371 & -0.389 & 1 & -0.445\\ \hline
	 & Sig. (2-tailed) & 0.297 & 0.013 & 0.143 & 0.970 & 0.046 & 0.267 & 0.379 & 0.025 & 0.635 & 0.055 & 0.151 & 0.291 & 0.267 &  & 0.197\\ \hline
	 & N & 10 & 10 & 10 & 10 & 10 & 10 & 10 & 10 & 10 & 10 & 10 & 10 & 10 & 10 & 10 \\ \hline
	Churn\_Average & Pearson Correlation & 0.223 & -0.575 & -0.624 & -0.471 & -0.319 & -.689* & .981** & .766** & .842** & -0.406 & -0.035 & 0.016 & .846** & -0.445 & 1 \\ \hline
	 & Sig. (2-tailed) & 0.537 & 0.082 & 0.054 & 0.170 & 0.369 & 0.027 & 0 & 0.010 & 0.002 & 0.245 & 0.925 & 0.966 & 0.002 & 0.197 &  \\ \hline
	 & N & 10 & 10 & 10 & 10 & 10 & 10 & 10 & 10 & 10 & 10 & 10 & 10 & 10 & 10 & 10 \\ \hline
\end{tabular}
      \end{tabular}%
    }
    \centerline{*Correlation is significant at the 0.05 level (2-tailed).}
  \caption{Krypton - correlation}%
  \label{swag}
    \end{adjustwidth}
\end{table}%

 \begin{table}[!htbp]
\centering
  \begin{adjustwidth}{-2cm}{}
     \scalebox{0.75}{
\begin{tabular}{ | l | l | l | l | l | }
\hline
	Team & Quarter &  & Leadtime & Churn \\ \hline
	Krypton & 2010-3 & N & 46 & 46 \\ \hline
	 &  & Mean & 7.89 & 159.38 \\ \hline
	 &  & Median & 3.5 & 26.5 \\ \hline
	 &  & Std. Deviation & 11.651 & 206.584 \\ \hline
	 &  & Maximum & 71 & 647 \\ \hline
	 &  & Minimum & 1 & 0 \\ \hline
	 & 2010-4 & N & 139 & 139 \\ \hline
	 &  & Mean & 7.38 & 97.7 \\ \hline
	 &  & Median & 3 & 28 \\ \hline
	 &  & Std. Deviation & 12.88 & 145.84 \\ \hline
	 &  & Maximum & 85 & 700 \\ \hline
	 &  & Minimum & 1 & 0 \\ \hline
	 & 2011-1 & N & 153 & 153 \\ \hline
	 &  & Mean & 5.83 & 71.38 \\ \hline
	 &  & Median & 2 & 17 \\ \hline
	 &  & Std. Deviation & 7.383 & 130.40\\ \hline
	 &  & Maximum & 32 & 726 \\ \hline
	 &  & Minimum & 1 & 0 \\ \hline
	 & 2011-2 & N & 51 & 51 \\ \hline
	 &  & Mean & 14.37 & 110.45 \\ \hline
	 &  & Median & 9 & 31 \\ \hline
	 &  & Std. Deviation & 16.09 & 165.053 \\ \hline
	 &  & Maximum & 90 & 719 \\ \hline
	 &  & Minimum & 1 & 0 \\ \hline
	 & 2011-3 & N & 49 & 49 \\ \hline
	 &  & Mean & 13.71 & 119.43 \\ \hline
	 &  & Median & 8 & 27 \\ \hline
	 &  & Std. Deviation & 15.97 & 159.39 \\ \hline
	 &  & Maximum & 88 & 604 \\ \hline
	 &  & Minimum & 1 & 1 \\ \hline
	 & 2011-4 & N & 66 & 66 \\ \hline
	 &  & Mean & 10.5 & 112.36 \\ \hline
	 &  & Median & 6 & 36 \\ \hline
	 &  & Std. Deviation & 11.975 & 164.935 \\ \hline
	 &  & Maximum & 57 & 675 \\ \hline
	 &  & Minimum & 1 & 1 \\ \hline
	 & 2012-1 & N & 45 & 45 \\ \hline
	 &  & Mean & 23.73 & 99.09 \\ \hline
	 &  & Median & 8 & 23 \\ \hline
	 &  & Std. Deviation & 40.47& 149.02 \\ \hline
	 &  & Maximum & 180 & 636 \\ \hline
	 &  & Minimum & 1 & 1 \\ \hline
	 & 2012-2 & N & 45 & 45 \\ \hline
	 &  & Mean & 23 & 109.07 \\ \hline
	 &  & Median & 7 & 47 \\ \hline
	 &  & Std. Deviation & 32.08 & 131.58 \\ \hline
	 &  & Maximum & 135 & 536 \\ \hline
	 &  & Minimum & 1 & 1 \\ \hline
	 & 2012-3 & N & 74 & 74 \\ \hline
	 &  & Mean & 6.92 & 82.39 \\ \hline
	 &  & Median & 3 & 37 \\ \hline
	 &  & Std. Deviation & 9.26 & 122.637 \\ \hline
	 &  & Maximum & 35 & 713 \\ \hline
	 &  & Minimum & 1 & 1 \\ \hline
	 & 2012-4 & N & 2 & 2 \\ \hline
	 &  & Mean & 1.5 & 44 \\ \hline
	 &  & Median & 1.5 & 44 \\ \hline
	 &  & Std. Deviation & 0.70 & 1.41 \\ \hline
	 &  & Maximum & 2 & 45 \\ \hline
	 &  & Minimum & 1 & 43 \\ \hline
	 & Total & N & 683 & 683 \\ \hline
	 &  & Mean & 10.55 & 98.6 \\ \hline
	 &  & Median & 4 & 29 \\ \hline
	 &  & Std. Deviation & 18.1 & 149.01 \\ \hline
	 &  & Maximum & 180 & 726 \\ \hline
	 &  & Minimum & 1 & 0 \\ \hline
\end{tabular}
 }
\caption{Descriptive statistic  - Lead-time and churn - Krypton}

\end{adjustwidth}
 \end{table}%
 
   \begin{table}[!htbp]
   \centering
 \begin{tabular}{ | l | l | l | l | l | l | l | }
\hline
	Quarter & N & Mean & Median & Std. Deviation & Maximum & Minimum \\ \hline
	2010-3 & 14.00 & 3.07 & 2.00 & 3.47 & 14.00 & 1.00\\ \hline
	2010-4 & 71.00 & 2.73 & 2.00 & 2.05 & 11.00 & 1.00\\ \hline
	2011-1 & 67.00 & 2.55 & 2.00 & 1.96 & 13.00 & 1.00\\ \hline
	2011-2 & 38.00 & 1.97 & 2.00 & 1.03 & 4.00 & 1.00\\ \hline
	2011-3 & 43.00 & 1.65 & 1.00 & 0.92 & 4.00 & 1.00\\ \hline
	2011-4 & 53.00 & 1.83 & 1.00 & 1.31 & 6.00 & 1.00\\ \hline
	2012-1 & 40.00 & 1.90 & 1.00 & 1.46 & 7.00 & 1.00\\ \hline
	2012-2 & 27.00 & 2.07 & 2.00 & 1.36 & 6.00 & 1.00\\ \hline
	2012-3 & 40.00 & 2.70 & 2.00 & 2.43 & 13.00 & 1.00\\ \hline
	2012-4 & 3.00 & 1.00 & 1.00 & 0 & 1.00 & 1.00\\ \hline
	Total & 396.00 & 2.26 & 2.00 & 1.82 & 14.00 & 1.00\\ \hline
	 \end{tabular}
  	  \caption{Frontend - Descriptive statistic - Throughput }%
\end{table}
\begin{table}[!htbp]
\begin{tabular}{ | l | l | l | l | l | l | l | l | }
\hline
TeamName & Quarter & N & Mean & Median & Std. Deviation & Maximum & Minimum \\ \hline
Krypton & 2010-3 & 25 & 14.12 & 14 & 5.093 & 27 & 9 \\ \hline
	 & 2010-4 & 92 & 23 & 23.5 & 7.819 & 48 & 12 \\ \hline
	 & 2011-1 & 90 & 14.76 & 13.5 & 5 & 26 & 6 \\ \hline
	 & 2011-2 & 91 & 12.75 & 12 & 4.64 & 22 & 3 \\ \hline
	 & 2011-3 & 92 & 17.2 & 16 & 6.476 & 32 & 8 \\ \hline
	 & 2011-4 & 92 & 17.47 & 17 & 6.32 & 30 & 4 \\ \hline
	 & 2012-1 & 91 & 14.35 & 15 & 3.911 & 24 & 7 \\ \hline
	 & 2012-2 & 91 & 16.53 & 14 & 6.03 & 34 & 10 \\ \hline
	 & 2012-3 & 92 & 23.83 & 20 & 10.387 & 43 & 7 \\ \hline
	 & 2012-4 & 67 & 1.69 & 0 & 3.016 & 11 & 0 \\ \hline
	 & Total & 823 & 16.11 & 15 & 8.43& 48 & 0 \\ \hline
\end{tabular}
\caption{Krypton - Descriptive statistic WIP}
\end{table} 	

\begin{table}[!htbp]
\begin{tabular}{ | l | l | l | l | l | l | l | l | }
\hline
TeamName & Quarter & N & Mean & Median & Std. Deviation & Maximum & Minimum \\ \hline
Krypton & 2010-3 & 17 & 1.76 & 2 & 0.83 & 3 & 1 \\ \hline
	&2010-4 & 28 & 1.71 & 1.5 & 0.85 & 4 & 1  \  \\ \hline
	 & 2011-1 & 39 & 3.1 & 2 & 2.51 & 12 & 1 \\ \hline
	 & 2011-2 & 26 & 2.04 & 2 & 1.28 & 5 & 1 \\ \hline
	 & 2011-3 & 25 & 1.64 & 1 & 1.14 & 5 & 1 \\ \hline
	 & 2011-4 & 26 & 2 & 2 & 1.35 & 7 & 1 \\ \hline
	 & 2012-1 & 26 & 1.85 & 1 & 1.617 & 9 & 1 \\ \hline
	 & 2012-2 & 19 & 2.52 & 2 & 2.48 & 11 & 1 \\ \hline
	 & 2012-3 & 31 & 2.16 & 2 & 1.44 & 6 & 1 \\ \hline
	 & Total & 241 & 2.12 & 2 & 1.69 & 12 & 1 \\ \hline
	 & 2010-3 & 20 & 2.65 & 2 & 3.54 & 17 & 1 \\ \hline
\end{tabular}
\caption{Krypton - Descriptive statistic - Bugs}
\end{table}

\begin{table}[!htbp]
\begin{tabular}{ | l | l | l | l | l | l | l | l | }
\hline
TeamName & Quarter & N & Mean & Median & Std. Deviation & Maximum & Minimum \\ \hline
Krypton & 2010-3 & 10 & 2.20 & 1.5 & 1.619 & 5 & 1 \\ \hline
	 & 2010-4 & 48 & 2.79 & 2 & 2.278 & 11 & 1 \\ \hline
	 & 2011-1 & 30 & 1.7 & 1 & 0.95 & 4 & 1 \\ \hline
	 & 2011-2 & 17 & 1.65 & 1 & 1.115 & 4 & 1 \\ \hline
	 & 2011-3 & 20 & 1.45 & 1 & 0.82& 4 & 1 \\ \hline
	 & 2011-4 & 31 & 1.48 & 1 & 0.89 & 4 & 1 \\ \hline
	 & 2012-1 & 14 & 1.71 & 1 & 1.139 & 5 & 1 \\ \hline
	 & 2012-2 & 11 & 1.73 & 1 & 0.90& 3 & 1 \\ \hline
	 & 2012-3 & 17 & 1.53 & 1 & 0.8 & 4 & 1 \\ \hline
	 & 2012-4 & 3 & 1 & 1 & 0 & 1 & 1 \\ \hline
	 & Total & 201 & 1.9 & 1 & 1.48 & 11 & 1 \\ \hline
\end{tabular}
\caption{Krypton - Descriptive statistic  - TP feature}
\end{table}

\begin{table}[!htbp]
\begin{tabular}{ | l | l | l | l | l | l | l | l | }
\hline
TeamName & Quarter & N & Mean & Median & Std. Deviation & Maximum & Minimum \\ \hline
	Krypton & 2010-3 & 6 & 3.5 & 2.5 & 3.2090& 9 & 1 \\ \hline
	 & 2010-4 & 34 & 1.76 & 2 & 0.89 & 4 & 1 \\ \hline
	 & 2011-1 & 45 & 2.67 & 2 & 2.20 & 13 & 1 \\ \hline
	 & 2011-2 & 28 & 1.68 & 2 & 0.77& 4 & 1 \\ \hline
	 & 2011-3 & 26 & 1.62 & 1 & 0.752 & 3 & 1 \\ \hline
	 & 2011-4 & 27 & 1.89 & 1 & 1.45 & 6 & 1 \\ \hline
	 & 2012-1 & 28 & 1.86 & 1 & 1.407 & 7 & 1 \\ \hline
	 & 2012-2 & 18 & 2.06 & 2 & 1.552 & 6 & 1 \\ \hline
	 & 2012-3 & 28 & 2.93 & 3 & 2.19 & 9 & 1 \\ \hline
	 & Total & 240 & 2.13 & 2 & 1.66 & 13 & 1 \\ \hline
\end{tabular}
\caption{Krypton - Descriptive statistic  - TP bugs}
\end{table}


 %%%%%%%%%%%%%%%%
\section{Neon}

\begin{table}[!htbp]
  \centering
   \begin{adjustwidth}{-2.5cm}{}
     \scalebox{0.38}{
\begin{tabular}{ | l | l | l | l | l | l | l | l | l | l | l | l | l | l | l | l | l | }
\hline
	Correlations &  &  &  &  &  &  &  &  &  &  &  &  &  &  &  &  \\ \hline
	 &  & WIP & Throughput & Bugs & Churn & TP\_feature & Churn\_Bugs & Churn\_union & Leadtime\_union & Churn\_feature & Tp\_bugs & Average\_Days\_Backlog\_Bugs & Bugs\_Churn\_average & Churn\_feature\_Average & Precent\_bugs\_fininshed & Churn\_Average \\ \hline
	WIP & Pearson Correlation & 1 & 0.491 & 0.278 & -0.446 & 0.380 & -0.446 & 0.316 & 0.096 & 0.362 & 0.508 & -0.178 & 0.257 & 0.337 & 0.367 & 0.223\\ \hline
	 & Sig. (2-tailed) &  & 0.150 & 0.436 & 0.196 & 0.279 & 0.196 & 0.373 & 0.791 & 0.304 & 0.134 & 0.622 & 0.473 & 0.341 & 0.297 & 0.537\\ \hline
	 & N & 10 & 10 & 10 & 10 & 10 & 10 & 10 & 10 & 10 & 10 & 10 & 10 & 10 & 10 & 10 \\ \hline
	Throughput & Pearson Correlation & 0.491 & 1 & .773** & -0.036 & .741* & 0.217 & -0.494 & -0.539 & -0.433 & .758* & 0.055 & 0.259 & -0.609 & .745* & -0.575\\ \hline
	 & Sig. (2-tailed) & 0.150 &  & 0.009 & 0.921 & 0.014 & 0.548 & 0.147 & 0.108 & 0.212 & 0.011 & 0.879 & 0.470 & 0.062 & 0.013 & 0.082\\ \hline
	 & N & 10 & 10 & 10 & 10 & 10 & 10 & 10 & 10 & 10 & 10 & 10 & 10 & 10 & 10 & 10 \\ \hline
	Bugs & Pearson Correlation & 0.278 & .773** & 1 & 0.259 & 0.534 & 0.398 & -0.555 & -0.581 & -0.478 & 0.271 & 0.005 & 0.296 & -0.571 & 0.498 & -0.624\\ \hline
	 & Sig. (2-tailed) & 0.436 & 0.009 &  & 0.471 & 0.111 & 0.254 & 0.096 & 0.078 & 0.162 & 0.449 & 0.988 & 0.407 & 0.085 & 0.143 & 0.054\\ \hline
	 & N & 10 & 10 & 10 & 10 & 10 & 10 & 10 & 10 & 10 & 10 & 10 & 10 & 10 & 10 & 10 \\ \hline
	Churn & Pearson Correlation & -0.446 & -0.036 & 0.259 & 1 & -0.219 & .791** & -0.522 & -0.449 & -0.517 & -0.110 & -0.174 & 0.172 & -0.485 & -0.014 & -0.471\\ \hline
	 & Sig. (2-tailed) & 0.196 & 0.921 & 0.471 &  & 0.543 & 0.006 & 0.122 & 0.192 & 0.126 & 0.762 & 0.631 & 0.634 & 0.155 & 0.970 & 0.170\\ \hline
	 & N & 10 & 10 & 10 & 10 & 10 & 10 & 10 & 10 & 10 & 10 & 10 & 10 & 10 & 10 & 10 \\ \hline
	TP\_feature & Pearson Correlation & 0.380 & .741* & 0.534 & -0.219 & 1 & 0.214 & -0.244 & -0.312 & -0.026 & 0.427 & -0.256 & -0.007 & -0.294 & .640* & -0.319\\ \hline
	 & Sig. (2-tailed) & 0.279 & 0.014 & 0.111 & 0.543 &  & 0.553 & 0.497 & 0.381 & 0.943 & 0.219 & 0.475 & 0.984 & 0.409 & 0.046 & 0.369\\ \hline
	 & N & 10 & 10 & 10 & 10 & 10 & 10 & 10 & 10 & 10 & 10 & 10 & 10 & 10 & 10 & 10 \\ \hline
	Churn\_Bugs & Pearson Correlation & -0.446 & 0.217 & 0.398 & .791** & 0.214 & 1 & -.707* & -.771** & -0.460 & 0.040 & -0.378 & -0.035 & -0.590 & 0.388 & -.689* \\ \hline
	 & Sig. (2-tailed) & 0.196 & 0.548 & 0.254 & 0.006 & 0.553 &  & 0.022 & 0.009 & 0.181 & 0.912 & 0.282 & 0.924 & 0.073 & 0.267 & 0.027\\ \hline
	 & N & 10 & 10 & 10 & 10 & 10 & 10 & 10 & 10 & 10 & 10 & 10 & 10 & 10 & 10 & 10 \\ \hline
	Churn\_union & Pearson Correlation & 0.316 & -0.494 & -0.555 & -0.522 & -0.244 & -.707* & 1 & .704* & .844** & -0.368 & -0.139 & 0.152 & .841** & -0.312 & .981** \\ \hline
	 & Sig. (2-tailed) & 0.373 & 0.147 & 0.096 & 0.122 & 0.497 & 0.022 &  & 0.023 & 0.002 & 0.296 & 0.702 & 0.674 & 0.002 & 0.379 & 0\\ \hline
	 & N & 10 & 10 & 10 & 10 & 10 & 10 & 10 & 10 & 10 & 10 & 10 & 10 & 10 & 10 & 10 \\ \hline
	Leadtime\_union & Pearson Correlation & 0.096 & -0.539 & -0.581 & -0.449 & -0.312 & -.771** & .704* & 1 & 0.492 & -0.301 & 0.327 & -0.227 & 0.631 & -.697* & .766** \\ \hline
	 & Sig. (2-tailed) & 0.791 & 0.108 & 0.078 & 0.192 & 0.381 & 0.009 & 0.023 &  & 0.148 & 0.398 & 0.357 & 0.528 & 0.050 & 0.025 & 0.010\\ \hline
	 & N & 10 & 10 & 10 & 10 & 10 & 10 & 10 & 10 & 10 & 10 & 10 & 10 & 10 & 10 & 10 \\ \hline
	Churn\_feature & Pearson Correlation & 0.362 & -0.433 & -0.478 & -0.517 & -0.026 & -0.460 & .844** & 0.492 & 1 & -0.314 & -0.358 & -0.201 & .921** & -0.172 & .842** \\ \hline
	 & Sig. (2-tailed) & 0.304 & 0.212 & 0.162 & 0.126 & 0.943 & 0.181 & 0.002 & 0.148 &  & 0.377 & 0.310 & 0.579 & 0 & 0.635 & 0.002\\ \hline
	 & N & 10 & 10 & 10 & 10 & 10 & 10 & 10 & 10 & 10 & 10 & 10 & 10 & 10 & 10 & 10 \\ \hline
	Tp\_bugs & Pearson Correlation & 0.508 & .758* & 0.271 & -0.110 & 0.427 & 0.040 & -0.368 & -0.301 & -0.314 & 1 & 0.096 & 0.034 & -0.393 & 0.622 & -0.406\\ \hline
	 & Sig. (2-tailed) & 0.134 & 0.011 & 0.449 & 0.762 & 0.219 & 0.912 & 0.296 & 0.398 & 0.377 &  & 0.792 & 0.926 & 0.261 & 0.055 & 0.245\\ \hline
	 & N & 10 & 10 & 10 & 10 & 10 & 10 & 10 & 10 & 10 & 10 & 10 & 10 & 10 & 10 & 10 \\ \hline
	Average\_Days\_Backlog\_Bugs & Pearson Correlation & -0.178 & 0.055 & 0.005 & -0.174 & -0.256 & -0.378 & -0.139 & 0.327 & -0.358 & 0.096 & 1 & -0.267 & -0.272 & -0.489 & -0.035\\ \hline
	 & Sig. (2-tailed) & 0.622 & 0.879 & 0.988 & 0.631 & 0.475 & 0.282 & 0.702 & 0.357 & 0.310 & 0.792 &  & 0.456 & 0.446 & 0.151 & 0.925\\ \hline
	 & N & 10 & 10 & 10 & 10 & 10 & 10 & 10 & 10 & 10 & 10 & 10 & 10 & 10 & 10 & 10 \\ \hline
	Bugs\_Churn\_average & Pearson Correlation & 0.257 & 0.259 & 0.296 & 0.172 & -0.007 & -0.035 & 0.152 & -0.227 & -0.201 & 0.034 & -0.267 & 1 & -0.197 & 0.371 & 0.016\\ \hline
	 & Sig. (2-tailed) & 0.473 & 0.470 & 0.407 & 0.634 & 0.984 & 0.924 & 0.674 & 0.528 & 0.579 & 0.926 & 0.456 &  & 0.585 & 0.291 & 0.966\\ \hline
	 & N & 10 & 10 & 10 & 10 & 10 & 10 & 10 & 10 & 10 & 10 & 10 & 10 & 10 & 10 & 10 \\ \hline
	Churn\_feature\_Average & Pearson Correlation & 0.337 & -0.609 & -0.571 & -0.485 & -0.294 & -0.590 & .841** & 0.631 & .921** & -0.393 & -0.272 & -0.197 & 1 & -0.389 & .846** \\ \hline
	 & Sig. (2-tailed) & 0.341 & 0.062 & 0.085 & 0.155 & 0.409 & 0.073 & 0.002 & 0.050 & 0 & 0.261 & 0.446 & 0.585 &  & 0.267 & 0.002\\ \hline
	 & N & 10 & 10 & 10 & 10 & 10 & 10 & 10 & 10 & 10 & 10 & 10 & 10 & 10 & 10 & 10 \\ \hline
	Precent\_bugs\_fininshed & Pearson Correlation & 0.367 & .745* & 0.498 & -0.014 & .640* & 0.388 & -0.312 & -.697* & -0.172 & 0.622 & -0.489 & 0.371 & -0.389 & 1 & -0.445\\ \hline
	 & Sig. (2-tailed) & 0.297 & 0.013 & 0.143 & 0.970 & 0.046 & 0.267 & 0.379 & 0.025 & 0.635 & 0.055 & 0.151 & 0.291 & 0.267 &  & 0.197\\ \hline
	 & N & 10 & 10 & 10 & 10 & 10 & 10 & 10 & 10 & 10 & 10 & 10 & 10 & 10 & 10 & 10 \\ \hline
	Churn\_Average & Pearson Correlation & 0.223 & -0.575 & -0.624 & -0.471 & -0.319 & -.689* & .981** & .766** & .842** & -0.406 & -0.035 & 0.016 & .846** & -0.445 & 1 \\ \hline
	 & Sig. (2-tailed) & 0.537 & 0.082 & 0.054 & 0.170 & 0.369 & 0.027 & 0 & 0.010 & 0.002 & 0.245 & 0.925 & 0.966 & 0.002 & 0.197 &  \\ \hline
	 & N & 10 & 10 & 10 & 10 & 10 & 10 & 10 & 10 & 10 & 10 & 10 & 10 & 10 & 10 & 10 \\ \hline
\end{tabular}
    }
    \centerline{* Correlation is significant at the 0.05 level (2-tailed).}
    \centerline{** Correlation is significant at the 0.01 level (2-tailed).}
      \caption{Neon - Correlation}
  \label{tab:addlabel}%
      \end{adjustwidth}
\end{table}%

\begin{table}[!htbp]
\centering
  \begin{adjustwidth}{-2cm}{}
     \scalebox{0.75}{
 \begin{tabular}{ | l | l | l | l | l | }
\hline
	Team & Quarter &  & Churn & Leadtime \\ \hline
	Neon & 2010-3 & N & 62 & 62 \\ \hline
	 &  & Mean & 25.27 & 7.69 \\ \hline
	 &  & Median & 9 & 6 \\ \hline
	 &  & Std. Deviation & 39.448 & 6.88 \\ \hline
	 &  & Maximum & 193 & 24 \\ \hline
	 &  & Minimum & 0 & 1 \\ \hline
	 & 2010-4 & N & 125 & 125 \\ \hline
	 &  & Mean & 33.13 & 6.66 \\ \hline
	 &  & Median & 15 & 5 \\ \hline
	 &  & Std. Deviation & 47.46 & 6.01 \\ \hline
	 &  & Maximum & 214 & 27 \\ \hline
	 &  & Minimum & 0 & 1 \\ \hline
	 & 2011-1 & N & 132 & 132 \\ \hline
	 &  & Mean & 23.81 & 6.53 \\ \hline
	 &  & Median & 9 & 5.5 \\ \hline
	 &  & Std. Deviation & 41.22 & 5.43 \\ \hline
	 &  & Maximum & 301 & 27 \\ \hline
	 &  & Minimum & 0 & 1 \\ \hline
	 & 2011-2 & N & 134 & 134 \\ \hline
	 &  & Mean & 36.1 & 10.71 \\ \hline
	 &  & Median & 7 & 6 \\ \hline
	 &  & Std. Deviation & 59.067 & 35.19 \\ \hline
	 &  & Maximum & 271 & 408 \\ \hline
	 &  & Minimum & 0 & 1 \\ \hline
	 & 2011-3 & N & 99 & 99 \\ \hline
	 &  & Mean & 40.86 & 8.34 \\ \hline
	 &  & Median & 17 & 6 \\ \hline
	 &  & Std. Deviation & 50.60 & 6.61 \\ \hline
	 &  & Maximum & 239 & 27 \\ \hline
	 &  & Minimum & 1 & 1 \\ \hline
	 & 2011-4 & N & 98 & 98 \\ \hline
	 &  & Mean & 63.31 & 8.5 \\ \hline
	 &  & Median & 38 & 7 \\ \hline
	 &  & Std. Deviation & 74.35 & 6.84 \\ \hline
	 &  & Maximum & 294 & 27 \\ \hline
	 &  & Minimum & 1 & 1 \\ \hline
	 & 2012-1 & N & 110 & 110 \\ \hline
	 &  & Mean & 56.75 & 5.16 \\ \hline
	 &  & Median & 21 & 4 \\ \hline
	 &  & Std. Deviation & 72.52 & 4.46 \\ \hline
	 &  & Maximum & 296 & 22 \\ \hline
	 &  & Minimum & 1 & 1 \\ \hline
	 & 2012-2 & N & 81 & 81 \\ \hline
	 &  & Mean & 44.95 & 5.05 \\ \hline
	 &  & Median & 22 & 4 \\ \hline
	 &  & Std. Deviation & 54.23 & 4.43 \\ \hline
	 &  & Maximum & 300 & 23 \\ \hline
	 &  & Minimum & 1 & 1 \\ \hline
	 & 2012-3 & N & 174 & 174 \\ \hline
	 &  & Mean & 66.12 & 5.78 \\ \hline
	 &  & Median & 33.5 & 4 \\ \hline
	 &  & Std. Deviation & 81.02 & 4.35\\ \hline
	 &  & Maximum & 299 & 21 \\ \hline
	 &  & Minimum & 1 & 1 \\ \hline
	 & 2012-4 & N & 120 & 120 \\ \hline
	 &  & Mean & 75.930000000000007 & 5.3 \\ \hline
	 &  & Median & 32 & 4.5 \\ \hline
	 &  & Std. Deviation & 86.49 & 4. \\ \hline
	 &  & Maximum & 303 & 17 \\ \hline
	 &  & Minimum & 1 & 1 \\ \hline
	 & Total & N & 1155 & 1155 \\ \hline
	 &  & Mean & 47.76 & 7 \\ \hline
	 &  & Median & 18 & 5 \\ \hline
	 &  & Std. Deviation & 66.37 & 13.112 \\ \hline
	 &  & Maximum & 303 & 408 \\ \hline
	 &  & Minimum & 0 & 1 \\ \hline
\end{tabular}
}
\caption{Descriptive statistic  - Lead-time and churn - Neon}

\end{adjustwidth}
 \end{table}%
 
   \begin{table}[!htbp]
\begin{tabular}{ | l | l | l | l | l | l | l | l | }
\hline
Team & Quarter & N & Mean & Median & Std. Deviation & Maximum & Minimum \\ \hline
 Neon & 2010-3 & 10 & 2 & 1.5 & 1.333 & 5 & 1 \\ \hline
	 & 2010-4 & 34 & 1.68 & 1 & 1.173 & 6 & 1 \\ \hline
	 & 2011-1 & 31 & 1.71 & 1 & 1.03 & 5 & 1 \\ \hline
	 & 2011-2 & 6 & 1.17 & 1 & 0.4 & 2 & 1 \\ \hline
	 & 2011-3 & 9 & 1 & 1 & 0 & 1 & 1 \\ \hline
	 & 2011-4 & 1 & 1 & 1 & . & 1 & 1 \\ \hline
	 & Total & 91 & 1.62 & 1 & 1.06 & 6 & 1 \\ \hline
\end{tabular}
\caption{Neon - Descriptive statistic - Throughput}
\end{table}
 
\begin{table}[!htbp]
\begin{tabular}{ | l | l | l | l | l | l | l | l | }
\hline
TeamName & Quarter & N & Mean & Median & Std. Deviation & Maximum & Minimum \\ \hline
Neon & 2010-3 & 25 & 14.4 & 15 & 6.19 & 23 & 6 \\ \hline
	 & 2010-4 & 92 & 21.41 & 20 & 7.17 & 41 & 9 \\ \hline
	 & 2011-1 & 90 & 27.2 & 27.5 & 4.90 & 38 & 17 \\ \hline
	 & 2011-2 & 91 & 29.71 & 27 & 14.42 & 62 & 12 \\ \hline
	 & 2011-3 & 92 & 31.98 & 30 & 8.90 & 55 & 18 \\ \hline
	 & 2011-4 & 92 & 29.09 & 29 & 10.09 & 45 & 12 \\ \hline
	 & 2012-1 & 91 & 19.03 & 18 & 4.63 & 30 & 7 \\ \hline
	 & 2012-2 & 91 & 24.34 & 25 & 10.282 & 50 & 5 \\ \hline
	 & 2012-3 & 92 & 23.24 & 21.5 & 7.89 & 44 & 10 \\ \hline
	 & 2012-4 & 92 & 19.48 & 22 & 10.99 & 45 & 1 \\ \hline
	 & 2013-1 & 11 & 2.18& 2 & 0.60 & 3 & 1 \\ \hline
	 & Total & 859 & 24.45 & 24 & 10.54& 62 & 1 \\ \hline
\end{tabular}
\caption{Neon - Descriptive statistic - WIP}
\end{table}

\begin{table}[!htbp]
\begin{tabular}{ | l | l | l | l | l | l | l | l | }
\hline
	TeamName & Quarter & N & Mean & Median & Std. Deviation & Maximum & Minimum \\ \hline
	Neon & 2010-4 & 40 & 2.45& 2 & 1.694 & 9 & 1 \\ \hline
	 & 2011-1 & 47 & 2.43 & 2 & 1.80& 8 & 1 \\ \hline
	 & 2011-2 & 42 & 3.57 & 3 & 2.52 & 13 & 1 \\ \hline
	&2011-3 & 45 & 2.47 & 2 & 2.33 & 13 & 1  \  \\ \hline
	 & 2011-4 & 48 & 2.48 & 2 & 1.59 & 7 & 1 \\ \hline
	 & 2012-1 & 36 & 3.25 & 3 & 3.03& 16 & 1 \\ \hline
	 & 2012-2 & 36 & 2.19 & 2 & 1.48 & 8 & 1 \\ \hline
	 & 2012-3 & 44 & 3.43 & 2 & 2.55 & 10 & 1 \\ \hline
	 & 2012-4 & 33 & 3.12 & 2 & 2.63& 10 & 1 \\ \hline
	 & 2013-1 & 1 & 1 & 1 & . & 1 & 1 \\ \hline
	 & Total & 404 & 2.75 & 2 & 2.306 & 17 & 1 \\ \hline
\end{tabular}
\caption{Neon - Descriptive statistic - Bugs}
\end{table}

\begin{table}[!htbp]
\begin{tabular}{ | l | l | l | l | l | l | l | l | }
\hline
	TeamName & Quarter & N & Mean & Median & Std. Deviation & Maximum & Minimum \\ \hline
	Neon & 2010-3 & 10 & 2 & 1.5 & 1.333 & 5 & 1 \\ \hline
	 & 2010-4 & 34 & 1.68 & 1 & 1.173 & 6 & 1 \\ \hline
	 & 2011-1 & 31 & 1.71 & 1 & 1.038 & 5 & 1 \\ \hline
	 & 2011-2 & 11 & 1.27 & 1 & 0.64 & 3 & 1 \\ \hline
	 & 2011-3 & 34 & 1.53 & 1 & 0.99 & 5 & 1 \\ \hline
	 & 2011-4 & 16 & 1.5 & 1 & 0.73 & 3 & 1 \\ \hline
	 & 2012-1 & 23 & 1.3 & 1 & 0.70 & 4 & 1 \\ \hline
	 & 2012-2 & 32 & 1.63 & 1 & 0.83 & 4 & 1 \\ \hline
	 & 2012-3 & 32 & 2.02& 2 & 0.93 & 4 & 1 \\ \hline
	 & 2012-4 & 40 & 1.93 & 2 & 1.11 & 6 & 1 \\ \hline
	 & Total & 263 & 1.69 & 1 & 0.997 & 6 & 1 \\ \hline
\end{tabular}
\caption{Neon - Descriptive statistic - TP feature}
\end{table}

\begin{table}[!htbp]
\begin{tabular}{ | l | l | l | l | l | l | l | l | }
\hline
	TeamName & Quarter & N & Mean & Median & Std. Deviation & Maximum & Minimum \\ \hline
	Neon & 2010-3 & 10 & 2 & 1.5 & 1.333 & 5 & 1 \\ \hline
	 & 2010-4 & 34 & 1.68 & 1 & 1.173 & 6 & 1 \\ \hline
	 & 2011-1 & 31 & 1.71 & 1 & 1.038 & 5 & 1 \\ \hline
	 & 2011-2 & 11 & 1.27 & 1 & 0.64 & 3 & 1 \\ \hline
	 & 2011-3 & 34 & 1.53 & 1 & 0.99 & 5 & 1 \\ \hline
	 & 2011-4 & 16 & 1.5 & 1 & 0.73 & 3 & 1 \\ \hline
	 & 2012-1 & 23 & 1.3 & 1 & 0.70 & 4 & 1 \\ \hline
	 & 2012-2 & 32 & 1.63 & 1 & 0.83 & 4 & 1 \\ \hline
	 & 2012-3 & 32 & 2.02& 2 & 0.93& 4 & 1 \\ \hline
	 & 2012-4 & 40 & 1.93 & 2 & 1.11 & 6 & 1 \\ \hline
	 & Total & 263 & 1.69 & 1 & 0.997 & 6 & 1 \\ \hline
\end{tabular}
\caption{Neon - Descriptive statistic - TP bugs}
\end{table}


\backmatter{}
\printbibliography
\end{document}
